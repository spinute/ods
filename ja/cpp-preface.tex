\cpponly{
\chapter*{C++ 版のまえがき}
\addcontentsline{toc}{chapter}{C++ 版のまえがき}

この本は基本的なデータ構造の設計と解析の手法、そしてオブジェクト指向言語での実装方法を教えるために書かれた。
具体的なオブジェクト指向言語としてはC++を採用している。

この本はC++を初めて学ぶ人のために書かれた本ではない。
一方で、C++あるいは似た言語の基本的な文法に馴染みがあれば、この本を読むには十分だ。
もし付属のソースコードにまで目を通したければ、C++でのプログラミング経験があったほうがいいだろう。

この本はC++の標準テンプレートライブラリ(STL)や、その背景にあるジェネリックプログラミングの手ほどきをするものでもない。
しかし、この本で実装を紹介するデータ構造の中には、STLでも使われているものも多く含まれている。
STLを使うプログラマは、STLではどうデータ構造を実装していて、それがなぜ効率的なのかを理解できるだろう。
}
