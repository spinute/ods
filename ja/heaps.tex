\chapter{ヒープ}
\chaplabel{heaps}
この章では優先度付きキューという役に立つデータ構造のふたつの実装を説明する。
いずれも特殊な二分木であり、\emph{ヒープ}(乱雑に積まれたもの)と呼ばれている。
\index{heap}%
\index{binary heap}%
\index{heap!binary}%
これは二分探索木が高度に構造化されながら積み上げられていたのとは対照的である。

ひとつめのヒープの実装は配列を使って完全二分木をシミュレートする。
この効率的な実装はヒープソート(\secref{heapsort}参照)という名の、最速の整列アルゴリズムのうちのひとつの基礎になっている。
ふたつめの実装はより柔軟である。
これはある優先度付きキューの要素を別の優先度付きキューに取り込む#meld(h)#操作を提供する。
