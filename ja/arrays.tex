\chapter{配列を使ったリスト}
\chaplabel{arrays}

この章では、\emph{backing array}と呼ばれる配列にデータを格納する、#List#・#Queue#インターフェースの実装について検討する。
\emph{backing array}.
\index{backing array}%
次の表は、この章で説明するデータ構造の操作時間を要約したものだ。

\newlength{\tabsep}
\setlength{\tabsep}{\itemsep}
\addtolength{\tabsep}{\parsep}
\addtolength{\tabsep}{-2pt}
\begin{center}
\vspace{\tabsep}
\begin{tabular}{|l|l|l|} \hline
 & #get(i)#/#set(i,x)# & #add(i,x)#/#remove(i)# \\ \hline
#ArrayStack# & $O(1)$ & $O(#n#-#i#)$ \\
#ArrayDeque# & $O(1)$ & $O(\min\{#i#,#n#-#i#\})$ \\
#DualArrayDeque# & $O(1)$ & $O(\min\{#i#,#n#-#i#\})$ \\
#RootishArrayStack# & $O(1)$ & $O(#n#-#i#)$ \\ \hline
\end{tabular}
\vspace{\tabsep}
\end{center}

データをひとつの配列に入れるデータ構造には以下のような利点・欠点がある。

\begin{itemize}
  \item 配列の任意の要素には一定の時間でアクセスできる。
  そのため#get(i)#・#set(i,x)#はいずれも定数時間で実行される。

  \item 配列は動的ではない。リストの真ん中付近に要素を追加・削除するためには、隙間を作ったり埋めたりするために多くの要素が移動することになる。
  #add(i,x)#・#remove(i)#操作の実行時間が#n#・#i#に依存するのはこのためだ。

  \item 配列は伸び縮みしない。
  backing arrayのサイズより多くの要素をデータ構造に入れるには、新しい配列を割当て、古い配列の要素をそちらにコピーする必要がある。
  この操作のコストは大きい。
\end{itemize}

3つめの点が重要だ。
上記の表に記載された実行時間はbacking arrayの拡大・縮小にかかるコストは含まれていない。
後述するように、注意深く設計すれば、backing arrayの拡大・縮小のコストを加味しても\emph{平均的な}実行時間はほとんど増えない。
より正確に言うと、空のデータ構造からはじめて、#add(i,x)#・#remove(i)#を#m#回実行するときの、backing arrayの拡大・縮小のための合計コストは$O(m)$である。
個々の操作のコストは大きいが、#m#個の操作にわたる償却コストを考えれば、ひとつの操作あたりのコストは$O(1)$なのだ。

\section{ArrayStack:配列を使った高速なスタック操作}
\seclabel{arraystack}

\index{ArrayStack@#ArrayStack#}%

#ArrayStack#は\emph{backing array}#a#を使ったリストインターフェースの実装だ。
リストの#i#番目の要素を#a[i]#とする。
ほとんどの場合#a#は実際に必要な値よりも大きい。
そのため整数#n#によって実際に#a#に入っている要素数を表す。
つまり、リストの要素は#a[0]#,\ldots,#a[n-1]#に格納される。
また、関係$#a.length# \ge #n#$が常に成り立つ。

\codeimport{ods/ArrayStack.a.n.size()}

\subsection{基本}
#get(i)#や#set(i,x)#を使って#ArrayStack#の要素を読み書きする方法は簡単である。
境界チェックをしたあと単に#a[i]#を返すか、#a[i]#を書き換えるかすればよいのだ。

\codeimport{ods/ArrayStack.get(i).set(i,x)}

#ArrayStack#に要素を追加・削除するための実装を\figref{arraystack}に示す。
#add(i,x)#では、まず#a#が既に一杯かどうかを調べる。
もしそうなら#resize()#を呼び出して、#a#を大きくする。
#resize()#の実装方法については後述する。
#resize()#の直後では$#a.length# > #n#$であることだけ知っていれば今は十分である。
あとは#x#が入るように$#a[i]#,\ldots,#a[n-1]#$をひとつずつ右に移動させ、#a[i]#をxにし、#n#を1増やせばよい。

\begin{figure}
  \begin{center}
    \includegraphics[scale=0.90909]{figs/arraystack}
  \end{center}
  \caption[Adding to an ArrayStack]{
  #add(i,x)#・#remove(i)#を#ArrayStack#に実行する。
  矢印は要素のコピーを表す。
  #resize()#を呼ぶ操作にはアスタリスクを付した。}
  \figlabel{arraystack}
\end{figure}

\codeimport{ods/ArrayStack.add(i,x)}
#resize()#が呼ばれるかもしれないが、このコストを無視すれば#add(i, x)#のコストは#x#を入れる場所を作るためにシフトする要素数に比例する。つまり、この操作のコストは(リサイズのことを無視すれば)$O(#n#-#i#)$である。

#remove(i)#の実装も似ている。
$#a[i+1]#,\ldots,#a[n-1]#$を左にひとつシフトし(#a[i]は書き換えられる))、#n#の値をひとつ小さくする。
その後#n#が#a.length#より小さすぎないか、具体的には$#a.length# \ge 3#n#$を確認する。
もしそうなら、#resize()#を呼んで#a#を小さくする。

\codeimport{ods/ArrayStack.remove(i)}
#resize()#が呼ばれるかもしれないが、このコストを無視すれば#remove(i)#のコストはシフトする要素数に比例し、$O(#n#-#i#)$である。

\subesction{拡張と収縮}

#resize()#の実装は単純だ。
大きさ$2#n#$の新しい配列#b#を割当て、#n#個の#a#の要素を#b#のはじめの#n#個としてコピーする。そして#a#を#b#に置き換える。
よって#resize()#の呼び出し後は$#a.length# = 2#n#$が成り立つ。

\codeimport{ods/ArrayStack.resize()}

#resize()#の実際のコストの計算も簡単だ。
大きさ$2#n#$の配列#b#を割当て、#n#要素をコピーする。
これは$O(#n#)$の時間がかかる。

前節からの実行時間分析は#resize()#のコストを無視していた。
この節では\emph{償却解析}として知られる方法でこれを解決する。
この手法は個々の#add(i, x)#・#remove(i)#における#resize()#のコストを求めるわけではない。
代わりに、#m#個の#add(i, x)#・#remove(i)#からなる操作列の間に呼ばれる#resize()#すべてのことを考える。

次の補題を示す。
\begin{lem}\lemlabel{arraystack-amortized}
  空の#ArrayStack#が作られたあと、$m\ge 1$個の#add(i, x)#・#remove(i)#からなる操作の列が順に実行されるとき、この間に呼ばれる#resize()#の合計実行時間は#O(m)#である。
\end{lem}

%XXX: ここまで完了
\begin{proof}
#resize()#が呼ばれるとき、その前の#resize()#呼び出しから#add#・#remove#が実行された回数は$#n#/2-1$以下である。
$i$回目の#resize()#呼び出しの際の#n#を$#n#_i$とする。
$r$を#resize()#の呼び出し回数とする。
このとき、#add(i,x)#と#remove(i)#の合計呼び出し回数は次の関係を満たす。
\[
  \sum_{i=1}^{r} (#n#_i/2-1) \le m \enspace
\]

これを変形すると次の式が得られる。
\[
  \sum_{i=1}^{r} #n#_i \le 2m + 2r  \enspace
\]

$r \leq m$より#resize()#呼び出しのための実行時間の合計は次のようになる。
\[
\sum_{i=1}^{r} O(#n#_i) \le O(m+r) = O(m)  \enspace
\]

あとは$(i-1)$から$i$回目の#resize()#の間に#add(i,x)#か#remove(i)#が呼ばれる回数が$#n#_i/2$以下であることを示す。

2つの場合が考えられる。
ひとつは#resize()#が#add(i,x)#に呼ばれる場合で、これはbacking arrayが一杯になるとき、つまり$#a.length# = #n#=#n#_i$が成り立つときだ。
直前の#resize()#を考えよう。この#resize()#の直後、#a#の大きさは#a.length#だが#a#の要素数は$#a.length#/2=#n#_i/2$以下であった。
しかし#a#の要素数は今では$#n#_i=#a.length#$なのだから、前の#resize()#から$#n#_i/2$回以上は#add(i,x)#が呼ばれたことがわかる。

もうひとつ考えられるのは、#resize()#が#remove(i)#に呼ばれる場合で、このとき$#a.length# \ge 3#n#=3#n#_i$である。
前の#resize()#の直後では#a#の要素数は$#a.length/2#-1$以下であった。
\footnote{この数式における${}-1$は、特別なケース$#n#=0$かつ$#a.length# = 1$を考慮したものだ。}
今#a#には$#n#_i\le#a.length#/3$個の要素が入っている。
よって、直前の#resize()#以降に実行された#remove(i)#の回数の下界は次のように計算できる。

  \begin{align*}
      R & \ge #a.length#/2 - 1 - #a.length#/3 \\
        & = #a.length#/6 - 1 \\
        & = (#a.length#/3)/2 - 1 \\
        & \ge #n#_i/2 -1\enspace .
  \end{align*}

いずれの場合でも、$(i-1)$から$i$回目の#resize()#の間に#add(i,x)#か#remove(i)#が呼ばれる回数の合計は$#n#_i/2-1$以上である。
\end{proof}

\subsection{要約}

次の定理は#ArrayStack#の性能を整理するものだ。

\begin{thm}\thmlabel{arraystack}
  #ArrayStack#は#List#インターフェースを実装する。
  #resize()#のコストを無視すると、#ArrayStack#における各操作の実行時間は、
  \begin{itemize}
    \item #get(i)#・#set(i,x)#の実行時間は$O(1)$である。
    \item #add(i,x)#・#remove(i)#の実行時間は$O(1+#n#-#i#)$である。
  \end{itemize}
  空の#ArrayStack#から任意の$m$個の#add(i,x)#・#remove(i)#からなる操作の列を実行する。このときすべての#resize()#にかかる時間の合計は$O(m)$である。
\end{thm}

#ArrayStack#は#Stack#を実装する効率的な方法である。
特に#push(x)#は#add(n,x)#、#pop()#は#remove(n-1)#のようにそれぞれ実装できる。
またいずれの操作も償却実行時間$O(1)$である。
