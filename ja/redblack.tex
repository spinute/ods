\chapter{赤黒木}
\chaplabel{redblack}

\index{binary search tree!red-black}%
% XXX: あかくろぎ?あかぐろき?あかぐろぎ? Wikipediaではあかくろぎだし、自分もそうだと思っています spinute % 標準語の語感だとあかくろぎでは。 YJ
\ejindex{red-black tree}{あかくろぎ@赤黒木}%
この章では赤黒木 (red-black tree)という、要素数に対して木の高さを対数程度に抑える二分木を紹介する。
赤黒木は最も広く使われるデータ構造のひとつである。
例えば、多くのライブラリの実装における基本的なデータ構造であり、JavaのコレクションフレームワークやC++の標準テンプレートライブラリ(のいくつかの実装)に使われている。
また、LinuxというOSのカーネルでも使われている。
赤黒木が人気な理由を挙げよう。
\begin{enumerate}
\item #n#個の要素を含む赤黒木の高さは$2\log #n#$以下である
\item #add(x)#・#remove(x)#を\emph{最悪の場合でも}$O(\log #n#)$の時間で実行できる
\item #add(x)#・#remove(x)#における、回転の回数は償却すると定数である
\end{enumerate}
はじめのふたつの性質が#Skiplist#・#Treap#・#Scapegoat#に対する赤黒木の優位性を示している。
#Skiplist#・#Treap#はランダム化を使うため実行時間$O(\log #n#)$は期待値にすぎない。
Scapegoat treeには高さの保証があるものの、#add(x)#・#remove(x)#の実行時間$O(\log #n#)$は償却実行時間にすぎない。
3つめの性質はおまけである。
この性質から、要素#x#を追加・削除するときの主な時間は、#x#を見つける処理によることがわかる。
\footnote{スキップリストや#Treap#も平均的にはこの性質を持つ。
\ref{exc:skiplist-changes}と\ref{exc:treap-rotates}を参照せよ。}

しかし、赤黒木におけるよい性質には代償がある。
実装が複雑なのである。
高さの上界を$2\log #n#$に保つのは容易ではない。
様々な可能性を、慎重に解析しなければならない。
そしてそのすべてにおいて、実装が正しくなければならないのである。
ひとつ回転を間違えたり、色を間違えると、わかりにくいバグが発生することになる。

赤黒木の実装に直接取り掛かるのでなく、まずは関連する背景知識として2-4木を説明する。
これは赤黒木が発見された経緯と、なぜ効率的に管理できるのかを理解する助けとなるだろう。

\section{2-4木}
\seclabel{twofour}

2-4木は次の性質を持つ根付き木である。
\begin{prp}[高さ]
すべての葉の深さは等しい。
\end{prp}
\begin{prp}[次数]
すべての内部ノードは2個以上4個以下の子ノードを持つ。
\end{prp}
2-4木の例を\figref{twofour-example}に示す。
\begin{figure}
  \begin{center}
    \includegraphics[scale=0.90909]{figs/24rb-2}
  \end{center}
  \caption{高さ3である2-4木}
  \figlabel{twofour-example}
\end{figure}
2-4木の性質より、この木の高さは葉の数の対数程度である。
\begin{lem}\lemlabel{twofour-height}
  #n#個の葉を持つ2-4木の高さは$\log #n#$以下である。
\end{lem}

\begin{proof}
内部ノードの子の数は2以上なので、2-4木の高さを$h$とすると葉の数は$2^h$以上である。
  \[
     #n# \ge 2^h \enspace .
  \]
  両辺の対数を取ると$h \le \log #n#$である。
\end{proof}

\subsection{葉の追加}
\seclabel{redblack-add-by-divide}

2-4木に葉を追加するのは簡単である。
(\figref{twofour-add}を参照せよ。)
他の葉の親ノード#w#の子として葉#u#を追加したいときは、単に#u#を#w#の子とする。
このとき高さの制約は保たれるが、次数の制約が成り立たなくなるかもしれない。
つまり、#u#を追加する直前に#w#の4つの子を持っていたなら、#w#の子の数は今では5となる。
この場合、 #w#を\emph{分割}し、2つの子を持つノード#w#と、3つの子を持つノード#w#'とする。
\ejindex{split}{ぶんかつ@分割}%
このとき#w#'には親がいないので、#w#の親を#w'#の親とする。
この処理は再帰的に行われる。
つまり、先の処理の結果として#w#の親が持つ子の数が多くなりすぎるかもしれず、その場合にはまた分割を行う。
この処理を、子の数が4未満のノードが見つかるか、根#r#を#r#と#r'#に分割するまで繰り返す。
後者の場合には新しい根を作り、#r#と#r'#をその子とする。
そのときにはすべての葉の深さが同時に増えるので、高さの性質はやはり保たれる
\footnote{訳注:これに似た議論は\secref{btree-elem-add}において現れる。}。

\begin{figure}
  \begin{center}
   \begin{tabular}{c}
     \includegraphics[scale=0.90909]{figs/24tree-add-1} \\
     \includegraphics[scale=0.90909]{figs/24tree-add-2} \\
     \includegraphics[scale=0.90909]{figs/24tree-add-3}
   \end{tabular}
  \end{center}
  \caption{2-4木に葉を追加する。
  #w.parent#の次数が4未満なので、この処理は1回の分割の後終了する。}
  \figlabel{twofour-add}
\end{figure}

2-4木の高さは常に$\log #n#$以下なので、葉の追加は$\log #n#$ステップ以下で完了する。

\subsection{葉の削除}

2-4木から葉を削除するには少し工夫が必要である。
(\figref{twofour-remove}を参照せよ。)
葉#u#をその親#w#から削除するとき、単に#u#を削除する。
その直前に、#w#がふたつしか子を持っていなければ、#w#の子はひとつだけになり、次数の制約を犯す。

\begin{figure}
  \begin{center}
   \begin{tabular}{c}
     \includegraphics[height=\FifthHeightScaleIfNeeded]{figs/24tree-remove-1} \\
     \includegraphics[height=\FifthHeightScaleIfNeeded]{figs/24tree-remove-2} \\
     \includegraphics[height=\FifthHeightScaleIfNeeded]{figs/24tree-remove-3} \\
     \includegraphics[height=\FifthHeightScaleIfNeeded]{figs/24tree-remove-4} \\
     \includegraphics[height=\FifthHeightScaleIfNeeded]{figs/24tree-remove-5} \\
   \end{tabular}
  \end{center}
  \caption{2-4木から葉を削除する。#u#の祖先とその兄弟はみな2つしか子を持っていないため、この処理は根まで繰り返される。}
  \figlabel{twofour-remove}
\end{figure}

これを修正するため、#w#の兄弟#w'#を見る。
#w#の親が持つ子の数は2以上なので、兄弟#w'#は必ず存在する。
#w'#が3つまたは4つの子を持つなら、そのうちひとつを#w'#から#w#に移す。
すると#w#の子の数は2、#w'#の子の数は2か3になり、処理を終えられる。

一方、#w'#の子の数が2なら、#w#と#w'#を\emph{併合}し、子を3つ持つ一つのノード#w#とする。
\ejindex{merge}{へいごう@併合}%
続いて#w'#を#w'#の親から取り除く。
この処理はノード#u#かその兄弟#u'#が3つ以上子を持つような#u#を見つけるか、根に到達すると終了する。
後者の場合、根はひとつの子だけを持つので、根は削除して、その子を新たな根とする。
この場合もすべての葉の高さが同時に減るので、高さの性質はやはり保たれる。

ここでも、2-4木の高さは常に$\log #n#$以下なので、葉の削除は$\log #n#$ステップ以下で完了する。

\section{#RedBlackTree#:2-4木のシミュレーション}
\seclabel{redblacktree}

赤黒木は各ノード#u#が\emph{赤}か\emph{黒}の\emph{色}を持つ二分探索木である。
\ejindex{colour}{いろ@色}%
赤は$0$で、黒は$1$で表現される。 % TODO: YJ: 赤ノード、黒ノードのほうが良いのでは。
\ejindex{red node}{あかいのーど@赤いノード}%
\ejindex{black node}{くろいのーど@黒いノード}%
\javaimport{ods/RedBlackTree.red.black.Node<T>}
\cppimport{ods/RedBlackTree.RedBlackNode.red.black}

赤黒木を操作する前後では次のふたつの性質が満たされる。
いずれも赤・黒の色と、0・1の数値を使って定義される。
\begin{prp}[黒の高さの性質] % こういう名前は英語の語感のものを日本語にそのまま訳するより日本語の語感に近い名前に変えた方がよいと思う。
  \ejindex{black-height property}{くろのたかさのせいしつ@黒の高さの性質}%
  「黒の高さ」が一様:葉から根への経路はいずれも、同じ数だけ黒いノードを含む。
\end{prp}

\begin{prp}[赤の辺の性質]
  \ejindex{no-red-edge property}{あかのへんのせいしつ@赤の辺の性質}%
  赤の辺が無い:赤いノード同士は隣接しない。
  (根でない任意のノード#u#について、$#u.colour# + #u.parent.colour# \ge 1$が成り立つ。)
\end{prp}
根#r#については、どちらの色であってもこれらの性質が満たされる。
そのためここでは、根は黒であると仮定する。
また赤黒木を更新するアルゴリズムはこれを保つようにする。
赤黒木を単純化するための別の工夫として、外部ノード(#nil#で表現される)を黒いノードと扱うのがよい。
\figref{redblack-example}に赤黒木の例を示した。

\begin{figure}
  \begin{center}
    \includegraphics[scale=0.90909]{figs/24rb-1}
  \end{center}
  \caption{黒の高さが3である赤黒木の例。外部ノード(#nil#)を正方形で描いている。}
  \figlabel{redblack-example}
\end{figure}


\subsection{赤黒木と2-4木}

前の小節で定義した黒の高さと赤の辺についての性質を保ちながら、赤黒木を効率的に更新できる事実には、はじめて聞くと驚くかもしれない。
一方で、これらの性質がなんの役に立つのかわからないかもしれない。
実は、赤黒木は2-4木を二分木として効率的にシミュレートするように設計されているのである。

\figref{twofour-redblack}を参照せよ。
#n#個のノードを持つ赤黒木$T$に次の変換を施す。
すべての赤いノード#u#を取り除き、#u#のふたつの子をいずれも(黒い)#u#の親に直接接続する。
こうして得られる木$T$は黒いノードだけを含む。
\begin{figure}
  \begin{center}
    \begin{tabular}{cc}
      \includegraphics[scale=0.90909]{figs/24rb-3} \\
      \includegraphics[scale=0.90909]{figs/24rb-2}
    \end{tabular}
  \end{center}
  \caption{任意の赤黒木には、対応する2-4木が存在する。}
  \figlabel{twofour-redblack}
\end{figure}

$T'$の内部ノードはみな2-4個の子を持つ。
ふたつの黒い子を持っていた黒いノードは、変換後も2つの黒い子を持つ。
赤い子と黒い子をひとつずつ持っていた黒いノードは、変換後は3つの黒い子を持つ。
ふたつの赤い子を持っていた黒いノードは、変換後は4つの黒い子を持つ。
加えて黒の高さの性質より、$T'$の任意の葉から根への経路の長さは同じである。
つまり、$T'$は2-4木なのである!

2-4木$T'$は$#n#+1$個の葉を持ち、各葉は赤黒木の$#n#+1$個の外部ノードと対応する。
よってこの木の高さは$\log (#n#+1)$以下である。
2-4木のすべての葉から根への経路は赤黒木$T'$における根から外部ノードへの経路に対応する。
この経路の最初・最後のノードは黒色で、内部ノードのふたつにひとつ以下が赤いので、この経路にあるノードのうち黒いものは$\log(#n#+1)$個以下、赤いものは$\log(#n#+1)-1$個以下である。
よって、任意の$#n#\ge 1$について、根から任意の\emph{内部}ノードへの経路のうち最長のものの長さは次の左辺の値以下である。
\[
   2\log(#n#+1) -2 \le 2\log #n#
\]
このことから、赤黒木の最も重要な性質を示せる。
\begin{lem}
#n#個のノードからなる赤黒木の高さは$2\log #n#$以下である。
\end{lem}

2-4木と赤黒木の関係がわかれば、赤黒木を保ちながら効率的に要素の追加・削除ができる気がしてきたことだろう。

#BinarySearchTree#における要素の追加は新たな葉を追加することで行えることはこれまでの章で説明した。
よって、赤黒木における#add(x)#を実装するためには、2-4木における5つの子を持つノードの分割をシミュレートする方法があればよい。
5つの子を持つ2-4木のノードは、ふたつの赤い子を持つひとつの黒いノード#w#であって、子のうちの一方が更に赤い子を持つものである。
#w#を「分割」するには、#w#を赤く、#w#の子をいずれも黒くすればよい。
例を\figref{rb-split}に示す。

\begin{figure}
  \begin{center}
   \begin{tabular}{c}
     \includegraphics[scale=0.90909]{figs/rb-split-1} \\
     \includegraphics[scale=0.90909]{figs/rb-split-2} \\
     \includegraphics[scale=0.90909]{figs/rb-split-3} \\
   \end{tabular}
  \end{center}
  \caption{2-4木における追加の際の分割を赤黒木で模倣する。(これは\figref{twofour-add}に示した2-4木への要素の追加を模倣している。)}
  \figlabel{rb-split}
\end{figure}

一方、#remove(x)#のためには、ふたつのノードを併合する方法と兄弟から子を借りる方法があればよい。
ふたつのノードの併合は\figref{rb-split}で示した分割の逆の処理であり、いずれも黒い兄弟を赤に、その共通の赤い親を黒にすればよい。
兄弟から子を借りる操作が最も複雑で、回転と色の変更を共に行う必要がある。

もちろん赤の辺の性質、黒の高さの性質をいずれも満たす必要がある。
これが可能なことくらいではもう驚かないかもしれないが、2-4木を赤黒木でシミュレートするために考慮すべき場合分けはやはり多い。
背景にある2-4木を無視して赤黒木の性質を保つことを直接的に考えることで、よりシンプルになることもある。

\subsection{左傾赤黒木}
\seclabel{left-redblack}
XXX: left-learning に定訳はあるか? % YJ: 政治の文脈では左傾と訳されるが、どうだろう。

% XXX: あかくろぎ?あかぐろき?あかぐろぎ? Wikipediaではあかくろぎだし、自分もそうだと思っています spinute
\ejindex{red-black tree}{あかくろぎ@赤黒木}%
\ejindex{left-leaning red-black tree}{left-learning あかくろぎ@left-learning 赤黒木}%
赤黒木の定義の仕方は複数ある。
#add(x)#・#remove(x)#を実行しながら、赤の辺の性質・黒の高さの性質を保てる、いくつかのデータ構造があるのだ。
異なる構造では、異なるやり方でこれを実現する。
この節では#RedBlackTree#と呼ぶデータ構造を実装する。
\index{RedBlackTree@#RedBlackTree#}%
これは赤黒木の一種であって、次の左傾性 (left-leaning) を満たすものである。
\begin{prp}[左傾性]\prplabel{left-leaning}\prplabel{redblack-last}
  \ejindex{left-leaning property}{left-learning せい@left-learning 性}%
  任意のノード#u#について、#u.left#が黒ならば#u.right#も黒である。
\end{prp}
例えば\figref{redblack-example}の左傾性を満たしていない。
右に向かう経路の赤いノードの親がこの性質を犯しているためだ。

左傾性を保持するのは、これにより#add(x)#・#remove(x)#において木を更新するときの場合分けが単純になるためである。
これは対応する2-4木の表現が一意に定まるためである。
次数が2のノードは黒いノードであって、ふたつの黒い子を持つものである。
次数が3のノードは黒いノードであって、左の子が赤く、右の子が黒の黒いものである。
次数が4のノードは黒いノードであって、ふたつの赤い子を持つものである。

#add(x)#・#remove(x)#の実装の詳細に入る前に、\figref{redblack-flippullpush}に示す単純なサブルーチンを導入する。
最初のふたつのサブルーチンは黒い高さの性質を保ったまま色を操作するものである。
#pushBlack(u)#の入力#u#はふたつの赤い子を持つ黒いノードで、#u#を赤に、そのふたつの子をいずれも黒に塗り替える。
#pullBlack(u)#はこの逆の操作である。
\codeimport{ods/RedBlackTree.pushBlack(u).pullBlack(u)}

\begin{figure}
  \begin{center}
    \includegraphics[width=\ScaleIfNeeded]{figs/flippullpush}
  \end{center}
  \caption{push・pull・flip}
  \figlabel{redblack-flippullpush}
\end{figure}

#flipLeft(u)#は#u#と#u.right#の色を入れ替え、その後#u#を左回転する。
この操作はこれらふたつのノードの色と親子関係をいずれも入れ替える。
\codeimport{ods/RedBlackTree.flipLeft(u)}
#flipLeft(u)#は#u#が左傾性を犯しているとき、この性質を取り戻すのに役立つ。
これは#u.left#が黒で#u.right#が赤であるためだ。
この場合は特に、この操作によって黒の高さ・赤の辺の性質がいずれも満たされることが保証される。
#flipRight(u)#は#flipLeft(u)#を左右対称に入れ替えた操作である。

XXX: 図のノードが一部白抜きなのはなぜ? % TODO: たぶん画像生成コードのバグかと。Patに確認しよう。
\codeimport{ods/RedBlackTree.flipRight(u)}

\subsection{要素の追加}

#RedBlackTree#において#add(x)#を実装するには、#BinarySearchTree#におけるふつうの挿入操作によって、$#u.x#=#x#$かつ$#u.colour#=#red#$を満たす新たな葉#u#を追加すればよい。
この処理ではどのノードの黒の高さも変わらないので、黒の高さの性質が犯されることはない。
しかし、左傾性が犯される可能性がある。(これは#u#が右の子であるときである。)また赤の辺の性質を犯すかもしれない。(これは#u#の親が赤いときである。)
これらの性質を取り戻すために、#addFixup(u)#を呼び出す。
\codeimport{ods/RedBlackTree.add(x)}

\figref{rb-addfix}に図示したように、#addFixup(u)#は赤いノード#u#を入力に取るが、これは赤の辺の性質や左傾性を犯しているかもしれない。
この先の議論は、\figref{rb-addfix}を見たり、それを一度紙に描いてみたりしなければ、理解するのが難しいかもしれない。
続きを読む前に、この図に目を通して意味を考えてみてほしい。

XXX: 白いノードの意味がよくわからない
\begin{figure}
  \begin{center}
    \includegraphics[width=\ScaleIfNeeded]{figs/rb-addfix}
  \end{center}
  \caption{要素を挿入したあと、ふたつの性質を満たすようにするための処理}
  \figlabel{rb-addfix}
\end{figure}

#u#が木の根なら、#u#を黒くすればふたつの性質が成り立つ。
#u#の兄弟も赤いなら、#u#の親は黒いのでふたつの性質は既に成り立っている。

このいずれでもないとき、まずは#u#の親#w#が左傾性を犯しているかを確認し、もしそうなら#flipLeft(w)#を実行し、$#u#=#w#$とする。
すると、#u#は親である#w#の左の子になり、そのため#w#は左傾性を満たすようになる。
あとは#u#の赤の辺の性質が満たされることを示せばよい。
#w#が黒いなら既に赤の辺の性質は満たされているので、赤い場合だけを心配すれば十分である。

まだ処理は終わりではなく、#u#と#w#はいずれも赤い。
赤い辺の性質(#u#によって侵されているが、#w#によっては侵されていない)より、#u#には親の親#g#が存在し、それは黒い。
#g#の右の子が赤いなら左傾性より#g#の子は共に赤く、#pushBlack(g)#を呼ぶと#g#は赤く、#w#は黒くなる。
すると、#u#について赤の辺の性質が満たされるが、#g#について赤の辺の性質が侵されるかもしれず、$#u#=#g#$として同じ処理を再度繰り返す。

もし#g#の右の子が黒いなら、
#flipRight(g)#を呼べば#w#は#g#の(黒い)親になり、#w#は#u#と#g#のふたつの赤い子を持つ。
これは#u#が赤い辺の性質を満たすこと、#g#が左傾性を満たすことを保証する。
この場合、処理はここで終了してよい。
\codeimport{ods/RedBlackTree.addFixup(u)}

#insertFixup(u)#の繰り返し一度あたりの実行時間は定数で、繰り返しの度に#u#を根に向けて動かすが処理が終了する。
よって、#insertFixup(u)#は$O(\log #n#)$回の繰り返しの後に終了し、このときの実行時間は$O(\log #n#)$である。

\subsection{要素の削除}
\seclabel{redblack-elem-remove}

#RedBlackTree#では#remove(x)#の実装が最も複雑であり、これは知られているどの赤黒木の場合でも同様である。
\texttt{二分探索木}における#remove(x)#のように、この操作は唯一の子#u#を持つノード#w#を特定し、#u#を#u.parent#と接続し、#w#を木から取り除く。

このとき#w#が黒いと、黒の高さの性質が#w.parent#で成り立たなくなってしまう。
#w.colour#を#u.colour#に足せば、この問題は一時的に解決する。
もちろんそうすると、ふたつの別の問題が発生しうる。
(1)~#u#と#w#が共に黒いとき、$#u.colour#+#w.colour#=2$であり、不正な色double-blackになってしまう。
(2)~#w#が赤いとき、#u#と入れ替えると$#u.parent#$の左傾性が犯されるかもしれない。
これらの問題はいずれも#removeFixup(u)#を呼ぶと解決できる。
\codeimport{ods/RedBlackTree.remove(x)}

#removeFixup(u)#の入力であるノード#u#の色は1か2(無効な色)である。
#u#の色が2なら、#removeFixup(u)#は回転と色の塗り替え操作とを繰り返し、double-blackのノードを木から追い出す。
この処理では、更新している部分木の根をノード#u#が参照するまで、更新を繰り返す。
この部分木の根の色は変わっているしれない。
具体的には赤から黒に変わっているかもしれず、そのため#removeFixup(u)#は最後に#u#の親がleft-learning性を満たしているか確認し、もし必要があればそれを修正する。
\codeimport{ods/RedBlackTree.removeFixup(u)}

\figref{rb-removefix}は#removeFixup(u)#を図示したものだ。
以降の説明も\figref{rb-removefix}をよく見てからでないと理解するのは難しいだろう。
#removeFixup(u)#の繰り返し毎に、double-blackであるノード#u#は、次の4つの場合分けに従って処理される。

\begin{figure}
  \begin{center}
    \includegraphics[height=\HeightScaleIfNeeded]{figs/rb-removefix}
  \end{center}
  \caption{削除のあと、double-blackであるノードを追い出す様子}
  \figlabel{rb-removefix}
\end{figure}

\noindent
Case 0: #u#が根である場合である。このときは最も簡単である。
単に#u#を黒に塗り直せばよい。(これは赤黒木のいずれの性質を犯すこともない。)

\noindent
Case 1: #u#の兄弟#v#が赤い場合である。
このとき左傾性より、#u#の兄弟#v#はその親#w#左の子である。
#w#で右回転を実行し、次の繰り返しに進む。
この操作では#w#の親は左傾性を犯すようになり、#u#の深さが1大きくなることに注意する。
しかし、次の繰り返しは#w#が赤い場合のCase~3である。
このとき、後で説明するCase~3を実行すれば、うまく処理できる。
\codeimport{ods/RedBlackTree.removeFixupCase1(u)}

\noindent
Case 2: #u#の兄弟#v#が黒い場合である。
このとき、#u#は親#w#の左の子である。
続いて#pullBlack(w)#を呼ぶ。すると、#u#は黒く、#v#は赤くなり、#w#はより黒く、すなわち黒またはdouble-blackになる。
このとき、#w#は左傾性を満たしておらず、#flipLeft(w)#を呼んでこれを解決する。

この時点で#w#は赤い。
#v#を処理を開始した部分木の根とする。
#w#が赤の辺の性質を犯していないか確認する必要がある。
このために、#w#の右の子#q#を見る。
#q#が黒いなら、#w#は赤の辺の性質を満たしており、$#u#=#v#$として次の繰り返しに進める。

そうでない(#q#が赤い)なら、#q#と#w#によって、赤の辺の性質と左傾性とが満たされていない。
左傾性を成り立たせるために、#rotateLeft(w)#を呼べばよい。
この時点で、赤の辺の性質は依然として犯されている。
#q#は#v#の左の子で、#w#は#q#の左の子である。
また、#q#と#w#はいずれも赤く、#v#は黒またはdouble-blackである。
#flipRight(v)#を実行すれば、#q#を#v#と#w#両方の親になる。
続いて、#pushBlack(q)#を呼ぶと#v#と#w#はいずれも黒くなり、#q#の色は#w#の元々の色になる。

すると、double-blackのノードは無くなり、赤の辺の性質・黒の高さの性質はいずれも満たされる。
あと気にする必要があるのは、#v#の右の子は赤いかもしれず、すると左傾性が犯されている。
そのためこれを確認し、もし必要なら#flipLeft(v)#を実行する。
\codeimport{ods/RedBlackTree.removeFixupCase2(u)}

\noindent
Case 3: #u#の兄弟が黒く、#u#が右の子である場合である。
この場合はCase~2と対称性があり、ほぼ同様に処理できる。
唯一の違いは、左傾性が非対称であることから生じる。
そのため、少し違った処理をする必要もある。

前と同様に、まずは#pullBlack(w)#によって#v#を赤く、#u#を黒くする。
#flipRight(w)#を呼ぶと#v#が部分木の根になる。
この時点で#w#は赤く、コードは#w#の左の子の色と#q#の色とに応じて分岐する。

#q#が赤いとき、Case~2の場合と全く同様に処理を終えることが出来る。
ただし、#v#が左傾性を満たさない心配はないので、より単純な処理で済ませられる。

より複雑なのは#q#が黒い場合の処理である。
この場合、#v#の左の子の色を確認する。
もし赤いなら#v#はふたつの赤い子を持っており、#pushBlack(v)#を実行して余分な黒を下に送ることができる。
この時点で#v#は#w#の元々の色になっており、処理を終えられる。

#v#の左の子が黒いなら、#v#は左傾性を犯しており、#flipLeft(v)#を呼んでこれを解消する。
そして、次の#removeFixup(u)#の繰り返しを$#u#=#v#$として続けるために、ノード#v#を返す。
\codeimport{ods/RedBlackTree.removeFixupCase3(u)}.

#removeFixup(u)#の各繰り返しは定数時間で実行できる。
Case~2・Case~3は処理を終了するか、#u#を根に近づける。
Case~0では(#u#が根であり)処理は常に終了する。
Case~1はCase~3に続き、その場合も処理は終了する。
木の高さは高々$2\log #n#$なので、高々$O(\log #n#)$回の#removeFixup(u)#を繰り返せばよいことがわかるので、#removeFixup(u)#の実行時間は$O(\log #n#)$である。

\section{要約}
\seclabel{redblack-summary}
次の定理は#RedBlackTree#の性能をまとめたものだ。

\begin{thm}
  #RedBlackTree#は#SSet#インターフェースの実装である。
  #RedBlackTree#は操作#add(x)#・#remove(x)#・#find(x)#を持ち、いずれの最悪実行時間も$O(\log #n#)$である。
\end{thm}

加えて次の定理も成り立つ。

\begin{thm}\thmlabel{redblack-amortized}
  空の#RedBlackTree#に対して、$m$個の#add(x)#・#remove(x)#からなる操作の列を実行するときの、#addFixup(u)#・#removeFixup(u)#に使われる時間の合計は$O(m)$である。
\end{thm}

\thmref{redblack-amortized}の証明は概要を示すだけにする。
#addFixup(u)#・#removeFixup(u)#と、2-4木における葉の追加・削除とを比べると、この性質は2-4木に由来するものという気がしてくるだろう。
特に2-4木における分割・併合・子を借りる処理に必要な時間が$O(m)$であることを示せれば、これから\thmref{redblack-amortized}を導けるだろう。

2-4木に関するこの定理の証明は、ポテンシャル法を使った償却解析による。
\ejindex{potential method}{ぽてんしゃるほう@ポテンシャル法}%
\footnote{ポテンシャル法の他の例としては、\lemref{dualarraydeque-amortized}・\lemref{selist-amortized}の証明を参照せよ。}
2-4木の内部ノード#u#のポテンシャルを次のように定義する。
\[
  \Phi(#u#) =
    \begin{cases}
      1 & \text{#u#の子の数が2のとき} \\ 
      0 & \text{#u#の子の数が3のとき} \\ 
      3 & \text{#u#のこの数が4のとき}
    \end{cases}
\]
また、2-4木のポテンシャルをそのすべてのノードのポテンシャルの和と定義する。
分割の際には、4つの子を持つノードがそれぞれ2つと3つの子を持つふたつのノードになる。
すなわち、全体のポテンシャルは$3-1-0 = 2$小さくなる。
併合の際には、それぞれふたつの子を持っていたふたつのノードが、3つの子を持つ一つのノードになる。
このとき、全体のポテンシャルは$2-0=2$小さくなる。
よって、分割や併合の際には、ポテンシャルは2だけ小さくなる。

分割と併合以外では、葉を加えたり削除したりして、子の数が変わるノードの個数は定数である。
ノードを追加するとき、ひとつのノードの子が1だけ増え、ポテンシャルは高々3増える。
ノードを削除するとき、ひとつのノードの子が1だけ減り、ポテンシャルは高々1増える。
また子を借りる処理にはふたつのノードが関連し、それらのポテンシャルは高々1だけ増える。

まとめると、分割・併合はポテンシャルを2以上減らす。
分割と併合以外では、追加と削除はポテンシャルを高々3増やし、ポテンシャルは常に非負である。
よって、空の木に対して$m$回の追加と削除を実行するとき、分割・併合は合わせて高々$3m/2$回だけ実行される。
\thmref{redblack-amortized}はこの解析の帰結であり、2-4木と赤黒木の間の対応を示している。

\section{ディスカッションと練習問題}

赤黒木はGuibasとSedgewick\cite{gs78}によってはじめに提案された。
赤黒木は実装が非常に複雑であるにも関わらず、ライブラリやアプリで最も頻繁に使われるもののうちの一つである。
ほとんどのアルゴリズムやデータ構造の本では何種類かの赤黒木を説明している。

Andersson \cite{a93}は左傾なバランスされた木であって、赤黒木に似ているが任意のノードは最大で一つの赤い子を持つという制約を加えたものを提案している。
そのためこのデータ構造がシミュレートするのは、2-4木ではなく2-3木である。
しかし、このデータ構造はこの章で説明した#RedBlackTree#よりもかなり単純である。

Sedgewick \cite{s08}は二種類の左傾赤黒木を提案している。
これらのデータ構造では、2-4木における上から下方向への分割・併合のシミュレーションに加えて、再帰を使っている。
こうすると、プログラムが短くエレガントになる。

関連するより古いデータ構造としては\emph{AVL木} \cite{avl62}がある。
\ejindex{AVL tree}{AVLぎ@AVL木}%
AVL木は次の\emph{height-balanced}性を満たす木である。
\ejindex{height-balanced}{たかさでばらんすされた@高さでバランスされた}%
\ejindex{binary search tree!height balanced}{にぶんたんさくぎ@二分探索木!たかさでばらんすされた@高さでバランスされた}%
任意のノード$u$について、#u.left#を根とする部分木の高さと、#u.right#を根とする部分木の高さとの差は、高々1である。
$F(h)$を高さ$h$である木のうち、葉が最も少ないものの葉の数とするとき、$F(h)$は次のフィボナッチの漸化式を満たす。
\[
   F(h) = F(h-1) + F(h-2)
\]
ただし$F(0)=1$, $F(1)=1$である。
ここで\emph{黄金比}$\varphi=(1+\sqrt{5})/2\approx1.61803399$とするとき、$F(h)$は近似的に$\varphi^h/\sqrt{5}$である。
(より正確には$|\varphi^h/\sqrt{5} - F(h)|\le 1/2$である。)
\lemref{twofour-height}の証明で述べたように、これは式を含意する。
\[
   h \le \log_\varphi #n# \approx 1.440420088\log #n#
\]
よって、AVL木の高さは赤黒木の高さよりも低い。
% XXX: rebalancing -> バランスの再調整
#add(x)#・#remove(x)#の際、根に向かって戻り、通過する各ノード#u#について、#u#の左右の部分木の高さが2以上異なるときバランスの再調整を行うことで高さのバランスを保つ。
\figref{avl-rebalance}を参照せよ。

\begin{figure}
  \begin{center}
    \includegraphics[scale=0.90909]{figs/avl-rebalance}
  \end{center}
  \caption{AVL木におけるバランスの再調整の様子。その子の部分木の大きさが$h$と$h+2$であるノードについて、いずれの子の部分木の高さも$h+1$とする際に必要な回転は高々2回である。}
  \figlabel{avl-rebalance}
\end{figure}

AnderssonのものやSedgewickのもの、AVL木のいずれも#RedBlackTree#よりも実装は単純である。
しかし、いずれもバランスの再調整のための償却実行時間が$O(1)$であることを保証できない。
特に\thmref{redblack-amortized}のような保証はない。

\begin{figure}
  \centering{\includegraphics[scale=0.90909]{figs/redblack-example}}
  \caption{練習問題のための赤黒木}
  \figlabel{redblack-example2}
\end{figure}

\begin{exc}
  \figref{redblack-example2}の#RedBlackTree#に対応する2-4木を図示せよ。
\end{exc}

\begin{exc}
  \figref{redblack-example2}の#RedBlackTree#に13, 3.5, 3.3を順に追加する様子を図示せよ。
\end{exc}

\begin{exc}
  \figref{redblack-example2}の#RedBlackTree#から11, 9, 5を順に削除する様子を図示せよ。
\end{exc}

\begin{exc}
どれだけ大きな#n#についても、#n#個のノードを持ち、高さが$2\log #n#-O(1)$である赤黒木が存在することを示せ。
\end{exc}

\begin{exc}
  操作#pushBlack(u)#・#pullBlack(u)#を考える。
  これらの操作は赤黒木によって表現されている2-4木に対して何を行うか。
\end{exc}

\begin{exc}
どれだけ大きな#n#についても、
どれだけ大きな#n#についても、#n#個のノードを持ち、高さが$2\log #n#-O(1)$である赤黒木を校正する#add(x)#・#remove(x)#からなる操作の列が存在することを示せ。
\end{exc}

\begin{exc}
#RedBlackTree#の実装における#remove(x)#で#u.parent=w.parent#とするのはなぜか説明せよ。
これは#splice(w)#において既に行われているはずではないか。
\end{exc}

\begin{exc}
2-4木$T$は$#n#_\ell$個の葉と$#n#_i$個の内部ノードを持つとする。
  \begin{enumerate}
    \item $#n#_\ell$が与えられたとき、$#n#_i$の最小値を求めよ。
    \item $#n#_\ell$が与えられたとき、$#n#_i$の最大値を求めよ。
    \item $T$を表現する赤黒木を$T'$とすると、$T'$の持つ赤いノードの個数を求めよ。
  \end{enumerate}
\end{exc}

\begin{exc}
#n#個のノードを持ち、高さが$2\log #n#-2$以下の二分探索木があるとする。
このとき、この木のすべてのノードを、黒の高さの性質と赤の辺の性質をいずれも満たすように、赤または黒に塗ることはできるか。
もしそうならそれに加えて左傾性を満たすことはできるか。
\end{exc}

\begin{exc}\exclabel{redblack-merge}
赤黒木$T_1$, $T_2$は黒の高さがいずれも$h$であり、$T_1$の最大のキーは$T_2$の最小のキーよりも小さいものであるとする。
このとき、$O(h)$の時間で$T_1$と$T_2$を一つの赤黒木に併合する方法を示せ。
\end{exc}

\begin{exc}
\excref{redblack-merge}の解法を拡張し、$T_1$, $T_2$の黒い高さ$h_1$, $h_2$が異なるとき、すなわち$h_1\neq h_2$であるときにも適用可能にせよ。
ただし、実行時間は$O(\max\{h_1,h_2\})$とする。
\end{exc}

\begin{exc}
AVL木における#add(x)#の間に、最大で一度だけバランスの再調整操作を実行しなければならないことを証明せよ。(このとき最大二度の回転を実行することになる。\figref{avl-rebalance}を参照せよ。)
また、#remove(x)#操作の際にオーダーで$\log #n#$だけのバランスの再調整操作が必要なAVL木の例を挙げよ。
\end{exc}

\begin{exc}
上で説明したAVL木の実装である#AVLTree#クラスを作成せよ。
この性能と#RedBlackTree#の性能とを比較せよ。
#find(x)#が高速に行えるのはどちらの実装か。
\end{exc}

\begin{exc}
#SSet#の実装である#SkiplistSSet#・#ScapegoatTree#・#Treap#・#RedBlackTree#における#find(x)#・#add(x)#・#remove(x)#の相対的な性能を評価する一連の実験を設計・実装せよ。
なお、ランダムなデータや整列済みのデータ、ランダム・規則正しい順序での削除などを含む、多くのテストのシナリオを作ること。
\end{exc}
