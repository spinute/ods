\chapter{赤黒木}
\chaplabel{redblack}

\index{binary search tree!red-black}%
\index{red-black tree}%
この章では赤黒木という高さを対数程度に抑える二分木を紹介する。
赤黒木は最も広く使われるデータ構造のうちのひとつである。
例えば、多くのライブラリの実装における基本的なデータ構造であり、JavaのコレクションフレームワークやC++の標準テンプレートライブラリ(のいくつかの実装)に使われている。
また、OSであるLinuxのカーネルにも使われている。
赤黒木が人気である理由を挙げる。
\begin{enumerate}
\item #n#個の値を持つ赤黒木の高さは$2\log #n#$以下である
\item #add(x)#・#remove(x)#を\emph{最悪の場合でも}$O(\log #n#)$の時間で実行できる
\item #add(x)#・#remove(x)#における、回転の回数は償却すると定数である
\end{enumerate}
はじめのふたつの性質が#Skiplist#・#Treap#・#Scapegoat#に対する赤黒木の優位性を示している。
#Skiplist#・#Treap#はランダム化を使うため実行時間$O(\log #n#)$は期待値にすぎない。
Scapegoat treeには高さの保証があるものの、#add(x)#・#remove(x)#の実行時間$O(\log #n#)$は償却実行時間にすぎない。
3つめの性質はおまけである。
要素#x#の追加・削除に必要な主要な時間は#x#を見つける処理によることを明らかにする。
\footnote{スキップリストや#Treap#も平均的にだがこの声質を持っている。
\ref{exc:skiplist-changes}と\ref{exc:treap-rotates}を参照せよ。}

しかし、赤黒木のよい性質には代償もある。
これは実装の複雑さである。
高さの上界を$2\log #n#$に保つのは容易ではない。
様々な場合についての慎重な解析が必要なのである。
すべての場合において、確実に正しい実行をしなければならないのである。
ひとつ回転を間違えたり、色を間違えると、わかりにくいバグが発生するのである。

赤黒木の実装に直接取り掛かるのでなく、関連するデータ構造である2-4木についての背景知識をまずは説明する。
こうするとどうやって赤黒木が発見され、なぜこのデータ構造を効率的に管理できるのかを理解する助けになるだろう。

\section{2-4木}
\seclabel{twofour}

2-4木は次の性質を持つ根付き木である。
\begin{prp}[height]
すべての葉の深さは同じである。
\end{prp}
\begin{prp}[degree]
すべての内部ノードは2-4個の子ノードを持つ。
\end{prp}
2-4の例を\figref{twofour-example}に示す。
\begin{figure}
  \begin{center}
    \includegraphics[scale=0.90909]{figs/24rb-2}
  \end{center}
  \caption{A 2-4 tree of height 3.}
  \figlabel{twofour-example}
\end{figure}
2-4木の性質より、この木の高さは葉の数の対数である。
\begin{lem}\lemlabel{twofour-height}
  #n#個の葉を持つ2-4木の高さは$\log #n#$以下である。
\end{lem}

\begin{proof}
内部ノードの子の数は2以上なので、2-4木の高さを$h$とすると葉の数は$2^h$以上である。
  \[
     #n# \ge 2^h \enspace .
  \]
  両辺の対数を取ると$h \le \log #n#$である。
\end{proof}

\subsection{葉の追加}

2-4木に葉を追加するのは簡単である。
(\figref{twofour-add}を参照せよ。)
下から二番目の深さのノード#w#の子として葉#u#を追加したいとき、単に#u#を#w#の子とする。
これは高さの制約は保つが、次数の制約を犯すかもしれない。
つまり、#u#を追加する直前に#w#の4つの子を持っていたなら、#w#の子の数は今では5となる。
この場合、 #w#を\emph{分割}し、#w#と#w#'というそれぞれ2つ、3つずつの子を持つノードとする。
\index{split}%
このとき#w#'には親がいないので、#w#'を#w#の親の子とする。
この処理は再帰的に行われる。
先の処理の結果として#w#の親が持つ子の数が多くなりすぎるかもしれず、そのときはまた分割を行う。
この処理を、子の数が4未満のノードが見つかるか、根#r#を#r#と#r'#に分割するまで繰り返す。
後者の場合には新しい根を作り、#r#と#r'#をその子とする。
そのときにはすべての葉の深さが同時に増えるので、高さの性質はやはり保たれる。

\begin{figure}
  \begin{center}
   \begin{tabular}{c}
     \includegraphics[scale=0.90909]{figs/24tree-add-1} \\
     \includegraphics[scale=0.90909]{figs/24tree-add-2} \\
     \includegraphics[scale=0.90909]{figs/24tree-add-3}
   \end{tabular}
  \end{center}
  \caption[Adding a leaf to a 2-4 Tree]{Adding a leaf to a 2-4 Tree.
  This process stops after one split because #w.parent# has a degree of less
  than 4 before the addition.}
  \figlabel{twofour-add}
\end{figure}

2-4木の高さは常に$\log #n#$以下なので、葉の追加は$\log #n#$ステップ以下で完了する。

\subsection{葉の削除}

2-4木から葉を削除するには少し工夫が必要である。
(\figref{twofour-remove}を参照せよ。)
葉#u#をその親#w#から削除するとき、単に#u#を削除する。
その直前に、#w#が子をふたつしか持っていなかったなら、#w#の子はひとつだけになり、次数の制約を犯すことになる。

\begin{figure}
  \begin{center}
   \begin{tabular}{c}
     \includegraphics[height=\FifthHeightScaleIfNeeded]{figs/24tree-remove-1} \\
     \includegraphics[height=\FifthHeightScaleIfNeeded]{figs/24tree-remove-2} \\
     \includegraphics[height=\FifthHeightScaleIfNeeded]{figs/24tree-remove-3} \\
     \includegraphics[height=\FifthHeightScaleIfNeeded]{figs/24tree-remove-4} \\
     \includegraphics[height=\FifthHeightScaleIfNeeded]{figs/24tree-remove-5} \\
   \end{tabular}
  \end{center}
  \caption[Removing a leaf from a 2-4 Tree]{Removing a leaf from a
    2-4 Tree.  This process goes all the way to the root because each of
    #u#'s ancestors and their siblings have only two children.}
  \figlabel{twofour-remove}
\end{figure}

これを修正するため、#w#の兄弟#w'#を見る。
#w#の親が持つ子の数は2以上なので、兄弟#w'#は必ず存在する。
#w'#が3つまたは4つの子を持つなら、そのうちひとつを#w'#から#w#に移す。
すると#w#の子の数は2、#w'#の子の数は2か3になり、処理を終えられる。

一方、#w'#の子の数が2なら、#w#と#w'#を\emph{併合}し、子を3つ持つ一つのノード#w#とする。
\index{merge}%
続いて#w'#を#w'#の親から取り除く。
この処理はノード#u#かその兄弟#u'#が3つ以上子を持つような#u#を見つけるか、根に到達すると終了する。
後者の場合は根はひとつの子だけを持つので、根は削除して、その子を新たな根とする。
この場合もすべての葉の高さが同時に減るので、高さの性質はやはり保たれる。

ここでも、2-4木の高さは常に$\log #n#$以下なので、葉の削除は$\log #n#$ステップ以下で完了する。
