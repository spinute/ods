\chapter{赤黒木}
\chaplabel{redblack}

\index{binary search tree!red-black}%
% XXX: あかくろぎ?あかぐろき?あかぐろぎ? Wikipediaではあかくろぎだし、自分もそうだと思っています spinute % 標準語の語感だとあかくろぎでは。 YJ / あかくろき、だと思ってました。。 -kshikano
\ejindex{red-black tree}{あかくろぎ@赤黒木}%
この章では、赤黒木(red-black tree)という、木の高さが要素数の対数で抑えられる二分木を紹介する。
赤黒木は最も広く使われるデータ構造のひとつである。
JavaのコレクションフレームワークやC++の標準テンプレートライブラリのいくつかの実装など、多くのライブラリで探索のための基本的なデータ構造として採用されている。
また、LinuxのOSカーネル内部でも使われている。
赤黒木が広く利用されている理由はいくつかある。
\begin{enumerate}
\item #n#個の要素を含む赤黒木の高さが$2\log #n#$以下になる
\item #add(x)#および#remove(x)#を\emph{最悪の場合でも} $O(\log #n#)$の時間で実行できる
\item #add(x)#および#remove(x)#における回転の回数は、償却すると定数である
\end{enumerate}
上記のうち、はじめの2つの性質が、#Skiplist#、#Treap#、#Scapegoat#に対する赤黒木の優位性を示している。
#Skiplist#と#Treap#ではランダム化を利用するので、実行時間$O(\log #n#)$は期待値にすぎない。
#Scapegoat#には高さに関する保証があるものの、#add(x)#と#remove(x)#の実行時間$O(\log #n#)$は償却実行時間にすぎない。
上記のうち3つめの性質は、おまけである。
この性質から、要素#x#を追加および削除するときにかかる主な時間は、#x#を見つける処理によることがわかる
\footnote{スキップリストや#Treap#も平均的にはこの性質を持つ。
\excref{skiplist-changes}および\excref{treap-rotates}参照。}。

しかし、赤黒木の優れた性質には代償がある。
実装が複雑なのである。
高さの上界を$2\log #n#$に保つのは容易ではない。
さまざまな可能性を慎重に解析する必要がある。
そして、そのすべてにおいて、実装が正しくなければならないのである。
回転を1つ間違えたり、色を間違えたりすると、わかりにくいバグが発生することになる。

ここでは、いきなり赤黒木の実装に取り掛からず、まずは関連する背景知識として2-4木について説明する。
これにより、赤黒木が発見された経緯と、なぜ赤黒木を効率的に管理できるのかを理解する助けとなるだろう。

\section{2-4木}
\seclabel{twofour}

2-4木は次の性質を持つ根付き木である。
\begin{prp}[高さ]
すべての葉の深さは等しい
\end{prp}
\begin{prp}[次数]
すべての内部ノードは2個以上4個以下の子ノードを持つ
\end{prp}
2-4木の例を\figref{twofour-example}に示す。
\begin{figure}
  \begin{center}
    \includegraphics[scale=0.90909]{figs/24rb-2}
  \end{center}
  \caption{高さ3である2-4木}
  \figlabel{twofour-example}
\end{figure}
上記の性質より、2-4木の高さは葉の数の対数で抑えられる。
\begin{lem}\lemlabel{twofour-height}
  #n#個の葉を持つ2-4木の高さは$\log #n#$以下である。
\end{lem}

\begin{proof}
内部ノードの子の数は2以上なので、2-4木の高さを$h$とすると葉の数は$2^h$以上である。
  \[
     #n# \ge 2^h
  \]
  両辺の対数を取ると$h \le \log #n#$である。
\end{proof}

\subsection{葉の追加}
\seclabel{redblack-add-by-divide}

2-4木に葉を追加するのは簡単である
(\figref{twofour-add})。
別の葉の親ノードである#w#の子として葉#u#を追加したいときは、単に#u#を#w#の子とする。
このとき、高さの制約は保たれるが、次数の制約が成り立たなくなるかもしれない。
つまり、#u#を追加する前に#w#が子を4つ持っていたなら、#w#の子の数が5となってしまう。
この場合、#w#を\emph{分割}し、子を2つ持つノード#w#と、子を3つ持つノード#w'#とする。
\ejindex{split}{ぶんかつ@分割}%
このとき#w'#には親がいないので、#w#の親を#w'#の親とする。
この処理を再帰的に繰り返す。
つまり、先の処理の結果として#w#の親が持つ子の数が多くなりすぎるかもしれないので、その場合には#w#の親を分割する。
この処理を、子の数が4未満のノードができるか、根#r#を#r#と#r'#に分割する状況に至るまで繰り返す。
後者の場合には新しい根を作り、#r#と#r'#をその新しい根の子とする。
そのときにはすべての葉の深さが同時に増えることになるので、高さの性質はやはり保たれる
\footnote{訳注:これに似た議論は\secref{btree-elem-add}において現れる。}。

\begin{figure}
  \begin{center}
   \begin{tabular}{c}
     \includegraphics[scale=0.90909]{figs/24tree-add-1} \\
     \includegraphics[scale=0.90909]{figs/24tree-add-2} \\
     \includegraphics[scale=0.90909]{figs/24tree-add-3}
   \end{tabular}
  \end{center}
  \caption{2-4木に葉を追加する。
  #w.parent#の次数が4未満なので、この処理は1回の分割の後終了する}
  \figlabel{twofour-add}
\end{figure}

2-4木の高さは常に$\log #n#$以下なので、葉の追加は$\log #n#$ステップ以下で完了する。

\subsection{葉の削除}

2-4木から葉を削除するには少し工夫が必要である
(\figref{twofour-remove})。
葉#u#をその親#w#から削除するときは、単に#u#を削除する。
その直前に#w#の子が2つだけだったなら、#w#の子が1つだけになるので、次数についての制約を満たさなくなる。

\begin{figure}
  \begin{center}
   \begin{tabular}{c}
     \includegraphics[height=\FifthHeightScaleIfNeeded]{figs/24tree-remove-1} \\
     \includegraphics[height=\FifthHeightScaleIfNeeded]{figs/24tree-remove-2} \\
     \includegraphics[height=\FifthHeightScaleIfNeeded]{figs/24tree-remove-3} \\
     \includegraphics[height=\FifthHeightScaleIfNeeded]{figs/24tree-remove-4} \\
     \includegraphics[height=\FifthHeightScaleIfNeeded]{figs/24tree-remove-5} \\
   \end{tabular}
  \end{center}
  \caption{2-4木から葉を削除する。#u#の祖先とその兄弟は、いずれも子が2つなので、この処理は根まで繰り返す}
  \figlabel{twofour-remove}
\end{figure}

これを修正するため、#w#の兄弟#w'#を調べる。
#w#の親が持つ子の数は2以上なので、兄弟#w'#は必ず存在する。
#w'#の子が3つまたは4つなら、そのうち1つを#w'#から#w#に移す。
すると、#w#の子は2つ、#w'#の子は2つもしくは3つになるので、処理を終える。

一方、#w'#の子が2つなら、#w#と#w'#を\emph{併合}して子を3つ持つノード#w#とする。
\ejindex{merge}{へいごう@併合}%
そのうえで、#w'#を#w'#の親から取り除く。
この処理は、自分自身もしくは兄弟が子を3つ以上持つようなノード#u#が見つかるか、根に到達したら終了する。
後者の場合、根の子は1つだけなので、その子を新たな根として以前の根を削除する。
この場合もすべての葉の高さが同時に減るので、高さの性質はやはり保たれる。

2-4木の高さは常に$\log #n#$以下なので、葉の削除もやはり$\log #n#$ステップ以下で完了する。

\section{#RedBlackTree#:2-4木をシミュレートする二分木}
\seclabel{redblacktree}

赤黒木は、各ノード#u#が\emph{赤}か\emph{黒}の\emph{色}を持つ二分探索木である。
\ejindex{colour}{いろ@色}%
赤ノードは$0$で、黒ノードは$1$で表現される。% TODO: YJ: 赤ノード、黒ノードのほうが良いのでは。 / いったんそちらに統一 -kshikano
\ejindex{red node}{あかのーど@赤ノード}%
\ejindex{black node}{くろのーど@黒ノード}%
\javaimport{ods/RedBlackTree.red.black.Node<T>}
\cppimport{ods/RedBlackTree.RedBlackNode.red.black}

赤黒木においては、すべての操作の前後で、次の2つの性質が満たされる。
どちらの性質も、赤および黒の色と、0および1の数値の、両方を使って定義される。
\begin{prp}[黒の高さの性質] % こういう名前は英語の語感のものを日本語にそのまま訳するより日本語の語感に近い名前に変えた方がよいと思う。
  \ejindex{black-height property}{くろのたかさのせいしつ@黒の高さの性質}%
  葉から根への経路には、いずれも黒ノードが同じ数だけ含まれる。(葉から根への経路について、色の総和はすべて等しい)
\end{prp}

\begin{prp}[赤の辺の性質]
  \ejindex{no-red-edge property}{あかのへんのせいしつ@赤の辺の性質}%
  赤の辺はない。すなわち、赤ノード同士は隣接しない。
  (根でない任意のノード#u#について、$#u.colour# + #u.parent.colour# \ge 1$が成り立つ。)
\end{prp}
これらの性質は、根#r#がどちらの色であっても満たされる。
そこで、ここでは根が黒であると仮定する。
また、赤黒木を更新するアルゴリズムでは、根は黒のまま保たれるものとする。
さらに、赤黒木を単純化するためのもう1つの工夫として、外部ノード(#nil#で表す)を黒ノードとして扱う。
\figref{redblack-example}に赤黒木の例を示す。

\begin{figure}
  \begin{center}
    \includegraphics[scale=0.90909]{figs/24rb-1}
  \end{center}
  \caption{黒の高さが3である赤黒木の例。正方形は外部ノード(#nil#)を表す}
  \figlabel{redblack-example}
\end{figure}


\subsection{赤黒木と2-4木}

赤黒木は、前項で定義した「黒の高さの性質」と「赤の辺の性質」を保ちながら効率的に更新できる。
この事実を初めて聞くと驚くことだろう。だが、そもそもこれらの性質がなんの役に立つのかわからない人もいるだろう。
実は、赤黒木は、2-4木を二分木として効率的にシミュレートするように設計されているのである。

\figref{twofour-redblack}を見てほしい。
#n#個のノードを持つ赤黒木$T$に対し、
「すべての赤ノード#u#を取り除き、#u#の2つの子を両方とも#u#の親(黒ノード)に直接接続する」という操作を実行する。
こうして得られる木$T'$は、黒ノードだけを含む。
\begin{figure}
  \begin{center}
    \begin{tabular}{cc}
      \includegraphics[scale=0.90909]{figs/24rb-3} \\
      \includegraphics[scale=0.90909]{figs/24rb-2}
    \end{tabular}
  \end{center}
  \caption{任意の赤黒木には、対応する2-4木が存在する}
  \figlabel{twofour-redblack}
\end{figure}

$T'$の内部ノードは、すべて2~4個の子を持つ。
黒い子を2つ持っていた黒ノードは、変換後も黒い子を2つ持つ。
赤い子と黒い子を1つずつ持っていた黒ノードは、変換後は黒い子を3つ持つ。
赤い子を2つ持っていた黒ノードは、変換後は黒い子を4つ持つ。
加えて、「黒の高さの性質」より、$T'$の任意の葉から根への経路の長さは同じである。
つまり、$T'$は2-4木なのである!

2-4木$T'$は$#n#+1$個の葉を持ち、各葉は赤黒木の$#n#+1$個の外部ノードと対応する。
よって、この木の高さは$\log (#n#+1)$以下である。
2-4木のすべての葉から根への経路は、赤黒木$T'$における根から外部ノードへの経路に対応する。
この経路の最初と最後のノードは黒で、2つの内部ノードのうち赤は1つ以下なので、この経路にあるノードのうち黒は$\log(#n#+1)$個以下、赤は$\log(#n#+1)-1$個以下である。
よって、任意の$#n#\ge 1$について、根から任意の\emph{内部}ノードへの経路のうちで最長のものの長さは、次の左辺の値以下である。
\[
   2\log(#n#+1) -2 \le 2\log #n#
\]
このことから、赤黒木の最も重要な次の性質を示せる。
\begin{lem}
#n#個のノードからなる赤黒木の高さは$2\log #n#$以下である。
\end{lem}

2-4木と赤黒木の関係がわかれば、赤黒木を維持したまま効率的に要素の追加や削除が可能であると思えてくるだろう。

これまでの章で、#BinarySearchTree#に対して要素を追加するには、新たな葉を追加すればいいことを見てきた。
よって、赤黒木における#add(x)#を実装するには、2-4木において子を5つ持つノードの分割をシミュレートする方法があればよい。
子を5つ持つ2-4木のノードは、赤い子を2つ持つ黒ノード#w#であって、2つの赤い子のうちの一方がさらに赤い子を持つようなものである。
#w#を「分割」するには、#w#を赤ノードにし、#w#の子をいずれも黒ノードにすればよい。
例を\figref{rb-split}に示す。

\begin{figure}
  \begin{center}
   \begin{tabular}{c}
     \includegraphics[scale=0.90909]{figs/rb-split-1} \\
     \includegraphics[scale=0.90909]{figs/rb-split-2} \\
     \includegraphics[scale=0.90909]{figs/rb-split-3} \\
   \end{tabular}
  \end{center}
  \caption{2-4木における追加の際の分割を赤黒木で模倣する(これは\figref{twofour-add}に示した2-4木への要素の追加を模倣している)}
  \figlabel{rb-split}
\end{figure}

#remove(x)#を実装するには、2つのノードを併合する方法と、兄弟から子を借りる方法があればよい。
2つのノードの併合は、\figref{rb-split}で示した分割とは逆の処理であり、黒い兄弟をいずれも赤ノードにし、その共通の赤い親を黒ノードにすればよい。
兄弟から子を借りる操作が最も複雑で、回転と色の変更が両方とも必要になる。

もちろん「赤の辺の性質」と「黒の高さの性質」をいずれも満たす必要がある。
それが可能なことは驚くに値しないだろうが、2-4木をシミュレートして赤黒木とするには、考慮すべき場合分けがかなり多くなる。
ある程度までいったら、背景となる2-4木は無視してしまい、赤黒木の性質を保つことだけを考えてシンプルな実装にする。

\subsection{左傾赤黒木}
\seclabel{left-redblack}
% XXX: left-learning に定訳はあるか? % YJ: 政治の文脈では左傾と訳されるが、どうだろう。

% XXX: あかくろぎ?あかぐろき?あかぐろぎ? Wikipediaではあかくろぎだし、自分もそうだと思っています spinute
\ejindex{red-black tree}{あかくろぎ@赤黒木}%
\ejindex{left-leaning red-black tree}{left-learning あかくろぎ@left-learning 赤黒木}%
赤黒木の定義は1つではない。
むしろ、#add(x)#と#remove(x)#の前後で「赤の辺の性質」と「黒の高さの性質」を維持できるようなデータ構造は複数ある。
データ構造が違えば、実装方法も異なる。
この節で実装するデータ構造を、#RedBlackTree#と呼ぶことにしよう。
\index{RedBlackTree@#RedBlackTree#}%
#RedBlackTree#は赤黒木の一種であり、左傾性(left-leaning)と呼ぶ次の性質を満たす。
\begin{prp}[左傾性]\prplabel{left-leaning}\prplabel{redblack-last}
  \ejindex{left-leaning property}{left-learning せい@left-learning 性}%
  任意のノード#u#について、#u.left#が黒ならば#u.right#も黒である。
\end{prp}
例えば、\figref{redblack-example}は左傾性を満たしていない。
右に向かう経路の赤ノードの親がこの性質を満たしていないからだ。

左傾性の維持に意味があるのは、これにより#add(x)#と#remove(x)#において木を更新するときの場合分けが単純になるからである。
なぜ単純になるかというと、対応する2-4木の表現が、次のように一意に定まるからである。
すなわち、2-4木における次数が2のノードは赤黒木における黒ノードであり、黒い子を2つ持つ。 % 「2-4木における」と明記した方が分かりやすいと思う
次数が3のノードは黒ノードであり、左の子が赤ノード、右の子が黒ノードである。
次数が4のノードは黒ノードであり、赤い子を2つ持つ。

#add(x)#と#remove(x)#の実装の詳細に入る前に、\figref{redblack-flippullpush}に示す単純なサブルーチンを導入する。
最初の2つのサブルーチンは、「黒の高さの性質」を保ったまま色を操作するものである。
#pushBlack(u)#の入力#u#は、赤い子を2つ持つ黒ノードで、#u#を赤ノードに、その2つの子をいずれも黒ノードに塗り替える。
#pullBlack(u)#は、その逆の操作である。
\codeimport{ods/RedBlackTree.pushBlack(u).pullBlack(u)}

\begin{figure}
  \begin{center}
    \includegraphics[width=\ScaleIfNeeded]{figs/flippullpush}
  \end{center}
  \caption{push、pull、flip}
  \figlabel{redblack-flippullpush}
\end{figure}

#flipLeft(u)#は、#u#と#u.right#の色を入れ替え、そのあとで#u#を左回転する。
この操作は、これら2つのノードの色と親子関係をいずれも入れ替えるのである。
\codeimport{ods/RedBlackTree.flipLeft(u)}
#flipLeft(u)#は、#u#が左傾性を満たしていないとき、すなわち#u.left#が黒ノード、#u.right#が赤ノードである場合に、左傾性を取り戻すのに役立つ。
この場合は特に、この操作によって、「黒の高さの性質」と「赤の辺の性質」がいずれも満たされることが保証される。
#flipRight(u)#は、#flipLeft(u)#を左右対称に入れ替えた操作である。

% XXX: 図のノードが一部白抜きなのはなぜ? % TODO: たぶん画像生成コードのバグかと。Patに確認しよう。
\codeimport{ods/RedBlackTree.flipRight(u)}

\subsection{要素の追加}

#RedBlackTree#において#add(x)#を実装するには、#BinarySearchTree#における通常の挿入操作によって、$#u.x#=#x#$かつ$#u.colour#=#red#$を満たす新たな葉#u#を追加すればよい。
この処理では、どのノードでも黒の高さは変わらないので、「黒の高さの性質」を満たさなくなることはない。
しかし、左傾性が満たされなくなる可能性がある(#u#が右の子である場合)。また、「赤の辺の性質」が満たされなくなることもある(#u#の親が赤ノードの場合)。
そこで、これらの性質を満たすようにするため、#addFixup(u)#を呼び出す。
\codeimport{ods/RedBlackTree.add(x)}

\figref{rb-addfix}に図示したように、#addFixup(u)#は赤ノード#u#を入力に取るが、これにより「赤の辺の性質」や左傾性を満たさなくなるかもしれない。
この先の議論を理解するには、\figref{rb-addfix}をよく観察したり、自分で紙に描いてみたりしないと難しいだろう。
続きを読む前に、\figref{rb-addfix}に目を通して意味を考えてみてほしい。

% XXX: 白いノードの意味がよくわからない
\begin{figure}
  \begin{center}
    \includegraphics[width=\ScaleIfNeeded]{figs/rb-addfix}
  \end{center}
  \caption{要素を挿入したあと、2つの性質を満たすようにするための処理}
  \figlabel{rb-addfix}
\end{figure}

#u#が木の根なら、#u#を黒ノードにすれば2つの性質が成り立つ。
#u#の兄弟も赤ノードなら、#u#の親は黒ノードなので、2つの性質はすでに成り立っている。

このいずれでもないとき、まずは#u#の親#w#が左傾性を満たしているか確認し、もし満たしていないなら#flipLeft(w)#を実行して$#u#=#w#$とする。
すると、#u#は親である#w#の左の子になるので、#w#が左傾性を満たすようになる。
あとは、#u#について「赤の辺の性質」が満たされることを示せばよい。
#w#が黒ノードなら、すでに「赤の辺の性質」は満たされているので、#w#が赤ノードである場合だけを心配すれば十分である。

まだ処理は終わりではない。いま、#u#と#w#はいずれも赤ノードである。
「赤い辺の性質」(#u#では満たされていないが、#w#では満たされている)より、#u#には親の親#g#が存在し、それは黒ノードである。
#g#の右の子が赤ノードなら、左傾性より、#g#の子は共に赤ノードである。#pushBlack(g)#を呼ぶと、#g#は赤ノードに、#w#は黒ノードになる。
これにより、#u#について「赤の辺の性質」が満たされるが、#g#については「赤の辺の性質」が満たされなくなっているかもしれないので、$#u#=#g#$として同じ処理を再度繰り返す。

もし#g#の右の子が黒ノードなら、
#flipRight(g)#を呼べば#w#は#g#の(黒い)親になり、#w#は#u#と#g#という2つの赤い子を持つ。
これは、#u#が「赤い辺の性質」を満たすこと、#g#が左傾性を満たすことを保証する。
この場合、処理はここで終了してよい。
\codeimport{ods/RedBlackTree.addFixup(u)}

#insertFixup(u)#は、繰り返しごとにかかる実行時間は定数で、繰り返しのたびに#u#を根のほうへ移動するか、もしくは処理が終了する。
よって、#insertFixup(u)#は$O(\log #n#)$回の繰り返しのあとに終了し、このときの実行時間は$O(\log #n#)$である。

\subsection{要素の削除}
\seclabel{redblack-elem-remove}

#RedBlackTree#の実装で最も複雑なのは#remove(x)#である。これは既知の赤黒木のすべてについていえる。
突き詰めて考えれば、\texttt{二分探索木}における#remove(x)#のように、#u#だけを子に持つノード#w#を特定し、#u#を#u.parent#と接続して、#w#を木から取り除くことになる。

このとき、#w#が黒ノードだと、「黒の高さの性質」が#w.parent#で成り立たなくなってしまう。
#w.colour#を#u.colour#に継ぎ足せば、この問題は一時的に解決する。
もちろん、そうすると、今度は次の2つの問題が発生する可能性がある。
(1)#u#と#w#が共に黒ノードのときは、$#u.colour#+#w.colour#=2$であり、不正な色(ダブルブラックと呼ぶことにする)になってしまう。
(2)#w#が赤ノードのときは、#u#と入れ替えることで、$#u.parent#$の左傾性が崩れるかもしれない。
これらの問題は、いずれも#removeFixup(u)#を呼ぶことで解決できる。
\codeimport{ods/RedBlackTree.remove(x)}

#removeFixup(u)#の入力であるノード#u#の色は、1もしくは2(ダブルブラック)である。
#u#の色がダブルブラックなら、#removeFixup(u)#によって回転と色の塗り替え操作を繰り返すことで、ダブルブラックのノードを木から追い出す。
この処理では、更新している部分木の根をノード#u#が参照するようになるまで、更新を繰り返す。
この部分木の根の色は変わっているかもしれない。
具体的には、赤から黒に変わっているかもしれない。そのため#removeFixup(u)#では、最後に#u#の親が左傾性を満たしているかどうかを確認し、必要があれば修正する。
\codeimport{ods/RedBlackTree.removeFixup(u)}

\figref{rb-removefix}に#removeFixup(u)#を図示する。
以降の説明も、\figref{rb-removefix}をよく観察しないと理解が難しいだろう。
#removeFixup(u)#を繰り返すたびに、ダブルブラックのノード#u#は、次の4つの場合分けに従って処理される。

\begin{figure}
  \begin{center}
    \includegraphics[height=\HeightScaleIfNeeded]{figs/rb-removefix}
  \end{center}
  \caption{削除のあと、double-blackであるノードを追い出す様子}
  \figlabel{rb-removefix}
\end{figure}

\noindent
Case 0:#u#が根である場合。このときは最も簡単で、
単に#u#を黒に塗り直せばよい(これは赤黒木のいずれの性質も崩さない)。

\noindent
Case 1:#u#の兄弟#v#が赤ノードの場合。
このときは、左傾性より、#u#の兄弟#v#はその親#w#の左の子である。
#w#で右回転を実行し、次の繰り返しに進む。
この操作では、#w#の親が左傾性を満たさなくなり、#u#の深さが1大きくなることに注意する。
しかし、次の繰り返しは#w#が赤ノードの場合(Case~3)である。
あとで説明するCase~3を実行すればうまく処理できる。
\codeimport{ods/RedBlackTree.removeFixupCase1(u)}

\noindent
Case 2:#u#の兄弟#v#が黒ノードの場合。
このとき、#u#は親#w#の左の子である。
#pullBlack(w)#を呼ぶと、#u#は黒ノードに、#v#は赤ノードになり、#w#は黒もしくはダブルブラックになる。
このとき#w#は左傾性を満たしておらず、#flipLeft(w)#を呼んでこれを解決する。

この時点で#w#は赤ノードであり、#v#は処理を開始した部分木の根となっている。
#w#が「赤の辺の性質」を満たしているか確認する必要がある。
そこで、#w#の右の子#q#を見る。
#q#が黒ノードなら、#w#は「赤の辺の性質」を満たしているので、$#u#=#v#$として次の繰り返しに進める。

そうでない(#q#が赤ノード)なら、#q#と#w#によって「赤の辺の性質」と左傾性が満たされていない。
左傾性を成り立たせるには、#rotateLeft(w)#を呼べばよい。
この時点で、「赤の辺の性質」は依然として崩れている。
#q#は#v#の左の子で、#w#は#q#の左の子である。
また、#q#と#w#はいずれも赤ノードであり、#v#は黒またはダブルブラックである。
#flipRight(v)#を実行すれば、#q#が#v#と#w#の両方の親になる。
続いて#pushBlack(q)#を呼ぶと、#v#と#w#はいずれも黒ノードになり、#q#の色は#w#のもともとの色になる。

これでダブルブラックのノードはなくなり、「赤の辺の性質」と「黒の高さの性質」がいずれも満たされる。
あと気にする必要があるのは、#v#の右の子は赤ノードかもしれないという点であり、その場合は左傾性が満たされなくなっている。
そのためこれを確認し、もし必要なら#flipLeft(v)#を実行する。
\codeimport{ods/RedBlackTree.removeFixupCase2(u)}

\noindent
Case 3:#u#の兄弟が黒ノードで、#u#が右の子である場合。
この場合はCase~2と対称であり、ほぼ同様に処理できる。
ただし、左傾性が非対称であることに起因する相違がある。
そのため少し違った処理をする必要もある。

前と同様に、まずは#pullBlack(w)#によって#v#を赤ノードに、#u#を黒ノードにする。
#flipRight(w)#を呼ぶと、#v#が部分木の根になる。
この時点で#w#は赤ノードである。そこで、#w#の左の子の色と#q#の色に応じて処理が分岐する。

#q#が赤ノードのときは、Case~2の場合とまったく同様に処理を終えられる。
ただし、#v#が左傾性を満たさなくなる心配がないので、さらに単純な処理で済ませられる。

複雑なのは#q#が黒ノードである場合の処理である。
この場合、#v#の左の子の色を確認する。
もし赤ノードなら、#v#の2つの子は赤ノードであり、#pushBlack(v)#を実行して余分な黒を下に送ることができる。
この時点で#v#は#w#のもともとの色になっているので処理を終えられる。

#v#の左の子が黒いなら、#v#は左傾性を満たさなくなっているので、#flipLeft(v)#を呼んでこれを解消する。
そして、次の#removeFixup(u)#の繰り返しを$#u#=#v#$として続けるために、ノード#v#を返す。
\codeimport{ods/RedBlackTree.removeFixupCase3(u)}

#removeFixup(u)#の各繰り返しは定数時間で実行できる。
Case~2とCase~3では、処理が終了するか、もしくは#u#を根に近づける。
Case~0では、#u#が根であり、処理は常に終了する。
Case~1は、すぐにCase~3を引き起こし、この場合も処理は終了する。
木の高さは高々$2\log #n#$なので、高々$O(\log #n#)$回の#removeFixup(u)#を繰り返せばよいことがわかる。したがって#removeFixup(u)#の実行時間は$O(\log #n#)$である。

\section{要約}
\seclabel{redblack-summary}
次の定理に#RedBlackTree#の性能をまとめる。

\begin{thm}
  #RedBlackTree#は#SSet#インターフェースの実装である。
  #RedBlackTree#には操作#add(x)#、#remove(x)#、#find(x)#があり、いずれの最悪実行時間も$O(\log #n#)$である。
\end{thm}

加えて次の定理も成り立つ。

\begin{thm}\thmlabel{redblack-amortized}
  空の#RedBlackTree#に対して、$m$個の#add(x)#および#remove(x)#からなる操作の列を実行するとき、#addFixup(u)#および#removeFixup(u)#に使われる時間の合計は$O(m)$である。
\end{thm}

\thmref{redblack-amortized}の証明は概要を示すだけとする。
#addFixup(u)#および#removeFixup(u)#と、2-4木における葉の追加および削除とを比べると、この性質は2-4木に由来するものであることが推察できるだろう。
特に、2-4木における分割、併合、子を借りる処理に必要な時間が$O(m)$であることを示せれば、これから\thmref{redblack-amortized}を導けるはずだ。

2-4木に関するこの定理の証明は、ポテンシャル法を使った償却解析による
\ejindex{potential method}{ぽてんしゃるほう@ポテンシャル法}%
\footnote{ポテンシャル法の他の例としては、\lemref{dualarraydeque-amortized}や\lemref{selist-amortized}の証明を参照せよ。}。
2-4木の内部ノード#u#のポテンシャルを次のように定義する。
\[
  \Phi(#u#) =
    \begin{cases}
      1 & \text{#u#の子の数が2のとき} \\ 
      0 & \text{#u#の子の数が3のとき} \\ 
      3 & \text{#u#の子の数が4のとき}
    \end{cases}
\]
また、2-4木のポテンシャルを、そのすべてのノードのポテンシャルの和と定義する。
分割の際には、4つの子を持つノードが、それぞれ2つの子と3つの子を持つ2つのノードになる。
すなわち、全体のポテンシャルは$3-1-0 = 2$だけ小さくなる。
併合の際には、それぞれ2つの子を持っていた2つのノードが、3つの子を持つ1つのノードになる。
このとき、全体のポテンシャルは$2-0=2$だけ小さくなる。
よって、分割や併合の際には、ポテンシャルは2だけ小さくなる。

分割と併合以外の処理については、葉を加えたり削除したりしたときに子の数が変わるノードの個数は定数である。
ノードを追加するときは、1つのノードの子が1だけ増え、ポテンシャルは高々3増える。
ノードを削除するときは、1つのノードの子が1だけ減り、ポテンシャルは高々1増える。
また、子を借りる処理には2つのノードが関連し、それらのポテンシャルは高々1だけ増える。

まとめると、分割と併合はポテンシャルを2以上減らす。
分割と併合以外では、追加と削除はポテンシャルを高々3増やし、ポテンシャルは常に非負である。
よって、空の木に対して$m$回の追加と削除を実行するとき、分割と併合は合わせて高々$3m/2$回だけ実行される。
\thmref{redblack-amortized}はこの解析の帰結であり、2-4木と赤黒木の間の対応を示している。

\section{ディスカッションと練習問題}

赤黒木はGuibasとSedgewickによって最初に提案された\cite{gs78}。
赤黒木は実装が非常に複雑であるにもかかわらず、ライブラリやアプリで最も頻繁に使われるデータ構造のうちの1つである。
アルゴリズムやデータ構造の本の多くでは、何種類かの赤黒木が説明されている。

Andersson \cite{a93}では、赤黒木に似た、左傾なバランスされた木が提案されている。
この木では、任意のノードが最大で1つの赤い子を持つ、という制約が加えられている。
そのため、このデータ構造がシミュレートするのは、2-4木ではなく2-3木である。
このデータ構造は、本章で説明した#RedBlackTree#よりもかなり単純である。

Sedgewick \cite{s08}では、二種類の左傾な赤黒木が提案されている。
それらのデータ構造では、2-4木における上から下方向への分割および併合のシミュレーションに加えて、再帰が利用されている。
これによりプログラムが短くエレガントになっている。

赤黒木に関連したより古いデータ構造として、\emph{AVL木} \cite{avl62}がある。
\ejindex{AVL tree}{AVLぎ@AVL木}%
AVL木は、\emph{height-balanced性}と呼ばれる性質を満たす木である。
\ejindex{height-balanced}{たかさでばらんすされた@高さでバランスされた}%
\ejindex{binary search tree!height balanced}{にぶんたんさくぎ@二分探索木!たかさでばらんすされた@高さでバランスされた}%
height-balanced性とは、「任意のノード$u$について、#u.left#を根とする部分木の高さと、#u.right#を根とする部分木の高さとの差は、高々1である」という性質だ。
この性質より、$F(h)$を高さ$h$である木のうちで葉が最も少ないものの葉の数とするとき、$F(h)$が次のフィボナッチの漸化式を満たすことが従う。
\[
   F(h) = F(h-1) + F(h-2)
\]
ただし$F(0)=1$, $F(1)=1$である。
ここで\emph{黄金比}を$\varphi=(1+\sqrt{5})/2\approx1.61803399$とするとき、$F(h)$は近似的に$\varphi^h/\sqrt{5}$である
(より正確には$|\varphi^h/\sqrt{5} - F(h)|\le 1/2$である)。
\lemref{twofour-height}の証明で述べたように、これは次の式を含意する。
\[
   h \le \log_\varphi #n# \approx 1.440420088\log #n#
\]
よって、AVL木の高さは赤黒木の高さよりも低い。
% XXX: rebalancing -> バランスの再調整
#add(x)#および#remove(x)#の際は、根に向かって戻る際に通過する各ノード#u#について、#u#の左右の部分木の高さが2以上異なる場合にはバランスを再調整して高さのバランスを保つ
(\figref{avl-rebalance})。

\begin{figure}
  \begin{center}
    \includegraphics[scale=0.90909]{figs/avl-rebalance}
  \end{center}
  \caption{AVL木におけるバランスの再調整の様子。その子の部分木の大きさが$h$と$h+2$であるノードについて、いずれの子の部分木の高さも$h+1$とする際に必要な回転は高々2回である}
  \figlabel{avl-rebalance}
\end{figure}

Anderssonのデータ構造、Sedgewickのデータ構造、AVL木は、いずれも#RedBlackTree#よりも実装が単純である。
しかし、いずれも、バランスを再調整するための償却実行時間が$O(1)$であることを保証できない。
特に、\thmref{redblack-amortized}のような保証はない。

\begin{figure}
  \centering{\includegraphics[scale=0.90909]{figs/redblack-example}}
  \caption{練習問題のための赤黒木}
  \figlabel{redblack-example2}
\end{figure}

\begin{exc}
  \figref{redblack-example2}の#RedBlackTree#に対応する2-4木を図示せよ。
\end{exc}

\begin{exc}
  \figref{redblack-example2}の#RedBlackTree#に13、3.5、3.3を順に追加する様子を図示せよ。
\end{exc}

\begin{exc}
  \figref{redblack-example2}の#RedBlackTree#から11、9、5を順に削除する様子を図示せよ。
\end{exc}

\begin{exc}
どれだけ大きな#n#についても、#n#個のノードを持ち、高さが$2\log #n#-O(1)$である赤黒木が存在することを示せ。
\end{exc}

\begin{exc}
  操作#pushBlack(u)#および#pullBlack(u)#を考える。
  これらは、赤黒木によって表現されている2-4木に対し、どんなことをする操作か。
\end{exc}

\begin{exc}
どれだけ大きな#n#についても、#n#個のノードを持ち、高さが$2\log #n#-O(1)$である赤黒木を構成する#add(x)#および#remove(x)#からなる操作の列が存在することを示せ。
\end{exc}

\begin{exc}
#RedBlackTree#の実装における#remove(x)#で、#u.parent=w.parent#とするのはなぜか説明せよ。
この処理は#splice(w)#においてすでに実行済みではないだろうか。
\end{exc}

\begin{exc}
2-4木$T$は、$#n#_\ell$個の葉と$#n#_i$個の内部ノードを持つとする。
  \begin{enumerate}
    \item $#n#_\ell$が与えられたとき、$#n#_i$の最小値を求めよ。
    \item $#n#_\ell$が与えられたとき、$#n#_i$の最大値を求めよ。
    \item $T$を表現する赤黒木を$T'$とすると、$T'$が持つ赤ノードの個数を求めよ。
  \end{enumerate}
\end{exc}

\begin{exc}
#n#個のノードを持ち、高さが$2\log #n#-2$以下の二分探索木があるとする。
このとき、この木のすべてのノードを、「黒の高さの性質」と「赤の辺の性質」をいずれも満たすように、赤または黒に塗ることはできるか。
もし可能なら、それに加えて左傾性を満たすことはできるか。
\end{exc}

\begin{exc}\exclabel{redblack-merge}
赤黒木$T_1$および$T_2$は、黒の高さがいずれも$h$であり、$T_1$の最大のキーは$T_2$の最小のキーよりも小さいとする。
このとき、$O(h)$の時間で$T_1$と$T_2$を1つの赤黒木に併合する方法を示せ。
\end{exc}

\begin{exc}
\excref{redblack-merge}の解法を拡張し、$T_1$および$T_2$の黒の高さ$h_1$および$h_2$が異なるとき、すなわち$h_1\neq h_2$であるときにも適用可能にせよ。
ただし、実行時間は$O(\max\{h_1,h_2\})$とする。
\end{exc}

\begin{exc}
AVL木における#add(x)#の間に、最大で一度だけバランスの再調整操作を実行しなければならないことを証明せよ(このとき、回転を最大で2回実行することになる。\figref{avl-rebalance}参照)。
また、#remove(x)#操作の際に必要となるバランス再調整のオーダーが$\log #n#$であるようなAVL木の例を挙げよ。
\end{exc}

\begin{exc}
上で説明したAVL木の実装として、#AVLTree#クラスを作成せよ。
作成した実装の性能と#RedBlackTree#の性能とを比較せよ。
#find(x)#が高速なのはどちらの実装か。
\end{exc}

\begin{exc}
#SSet#の実装である#SkiplistSSet#、#ScapegoatTree#、#Treap#、#RedBlackTree#における#find(x)#、#add(x)#、#remove(x)#の相対的な性能を評価する実験を設計、実装せよ。
ランダムなデータや整列済みのデータ、ランダムな順序や規則正しい順序での削除などを含む、さまざまなテストシナリオを作ること。
\end{exc}
