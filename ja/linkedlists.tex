\chapter{連結リスト}
\chaplabel{linkedlists}

\index{linked list}%
この章でも#List#インターフェースの実装を扱うが、次は配列ではなくポインタを使ったデータ構造の話をする。
この章のデータ構造は、要素を含むノードから構成される。
参照(ポインタ)を使ってノードを繋げて列を作る。
まずは単方向連結リストを紹介する。
これを使って#Stack#・(FIFO)#Queue#の操作を定数時間で実行できる。
次に二重連結リストを紹介する。
これを使うと#Deque#の操作を定数時間で実行できる。

連結リストを使って#List#インターフェースの実装するのは、配列を使う場合と比べて長所・短所がある。
どんな要素の#get(i)#・#set(i,x)#も定数時間で行えるわけではないのが主な短所だ。
その代わりに#i#番目の要素までリストをひとつずつ辿らなければならないのである。
一方でより動的であることが主な長所だ。
ノードの参照#u#があれば、#u#を削除したり、#u#の隣にノードを挿入したりを定数時間で実行できる。
これが#u#がリストの中のどのノードであっても成り立つのだ。
