\chapter{グラフ}
\chaplabel{graphs}

%\textbf{Warning to the Reader:} This chapter is still being actively
%developed, meaning that the code has not been thoroughly tested and/or
%the text has not be carefully proofread.

この章ではグラフのふたつの表現方法を説明し、それらを使う基本的なアルゴリズムを紹介する。

数学的には、\emph{(有向)グラフ}とは
\index{graph}%
\index{directed graph}%
組み$G=(V,E)$である。
ここで$V$は\emph{頂点}の集合であり、
\index{vertex}%
$E$は\emph{辺}と呼ばれる頂点の組みの集合である。
\index{edge}%
辺#(i,j)#は#i#から#j#に向いている。
\index{directed edge}%
XXX: source と target の訳語
#i#は辺の\emph{source}と呼ばれ、
\index{source}
#j#は\emph{target}と呼ばれる。
\index{target}
$G$における\emph{経路}%
\index{path}
とは頂点の列$v_0,\ldots,v_k$であって 任意の$i\in\{1,\ldots,k\}$について辺$(v_{i-1},v_{i})$が$E$に含まれるものである。
経路$v_0,\ldots,v_k$が
\emph{循環}である(循環している)とは、
\index{cycle}%
$(v_k,v_0)$も$E$の要素であることをいう。
経路(または循環)が\emph{単純}であるとは、
\index{simple path/cycle}%
経路に含まれる頂点が互いに異なることをいう。
頂点$v_i$から頂点$v_j$への経路があるとき、
$v_j$は$v_i$から\emph{到達可能}であるという。
\index{reachable vertex}
\figref{graph}にグラフの例を示した。

\begin{figure}
  \begin{center}
    \includegraphics[scale=0.90909]{figs/graph}
  \end{center}
  \caption{A graph with twelve vertices.  Vertices are drawn as numbered
    circles and edges are drawn as pointed curves pointing from source
    to target.}
  \figlabel{graph}
\end{figure}

グラフは多くの現象をモデル化できるので多くの応用を持つ。
自明な例がいくつかある。
コンピュータのネットワークはコンピュータを頂点、それらを繋ぐ(直接の)通信路を辺と見なせばグラフとしてモデル化できる。
街道は交差点を頂点、それらを繋ぐ通りを辺と見なせばグラフとしてモデル化できる。

もうすこし巧みな例は、グラフが集合における二項関係のモデルであることに着目すると見つかる。
例えば大学の時間割りにおける\emph{衝突グラフ}を考えられる。
\index{conflict graph}%
ここで頂点は大学の講義で、辺#(i,j)#は#i#と#j#の両方を受講する生徒がいることを表している。
よってこの辺から講義#i#・#j#のテストは同じ時間に割当てられてはならないことがわかる。

この節を通じて#n#は頂点の数を、#m#は辺の数を表すことにする。
すなわち$#n#=|V|$かつ$#m#=|E|$である。
さらに$V=\{0,\ldots,#n#-1\}$と仮定する。
他のデータを扱いたければ、大きさ#n#の配列にデータを入れて於けば良い。

グラフに対する典型的な操作は次のものだ。
\begin{itemize}
  \item #addEdge(i,j)#:辺$(#i#,#j#)$を$E$に加える。
  \item #removeEdge(i,j)#:辺$(#i#,#j#)$を$E$から除く。
  \item #hasEdge(i,j)#:$(#i#,#j#)\in E$ かどうかを調べる。
  \item #outEdges(i)#:$(#i#,#j#)\in E$を満たす整数整数$#j#$のリストを返す。
  \item #outEdges(i)#:$(#j#,#i#)\in E$を満たす整数整数$#j#$のリストを返す。
\end{itemize}

これらの操作を効率的に実装するのはさほど難しくない。
例えばはじめの3つの操作は#USet#を使って実装でき、\chapref{hashing}で説明したハッシュテーブルを使えば期待実行時間は定数である。
最後のふたつの操作は頂点を隣接行列のリストに入れれば定数時間で実行できる。

しかし、グラフにおける応用によって、各操作への要求が異なり、理想的にはこの要求をすべて満たす中で最も単純な実装を使いたい。
そのため、ふたつのグラフの表現方法のカテゴリについての説明をする。

\section{#AdjacencyMatrix#:行列によるグラフの表現}
\seclabel{adjacency-matrix}

\index{adjacency matrix}%
\emph{隣接行列}は#n#個の頂点を持つグラフ$G=(V,E)$を、各エントリが真偽値である$#n#\times#n#$行列#a#を使って表現したものである。
\codeimport{ods/AdjacencyMatrix.a.n.AdjacencyMatrix(n0)}

行列のエントリ#a[i][j]#は次のように定義される。
\[  #a[i][j]#=
    \begin{cases}
      #true# & \text{if $#(i,j)#\in E$} \\
      #false# & \text{otherwise}
    \end{cases}
\]
\figref{graph}のグラフの隣接行列を\figref{graph-adj}に示した。

この表現における#addEdge(i,j)#・#removeEdge(i,j)#・#hasEdge(i,j)#はいずれもエントリ#a[i][j]#を読み書きすればよい。
\codeimport{ods/AdjacencyMatrix.addEdge(i,j).removeEdge(i,j).hasEdge(i,j)}
これらの操作は明らかに定数時間で実行できる。

\begin{figure}
  \begin{center}
    \includegraphics[scale=0.90909]{figs/graph} \\[3ex]
    \begin{tabular}{c|cccccccccccc}
        &0&1&2&3&4&5&6&7&8&9&10&11 \\\hline
       0&0&1&0&0&1&0&0&0&0&0&0 &0\\
       1&1&0&1&0&0&1&1&0&0&0&0 &0\\
       2&1&0&0&1&0&0&1&0&0&0&0 &0\\
       3&0&0&1&0&0&0&0&1&0&0&0 &0\\
       4&1&0&0&0&0&1&0&0&1&0&0 &0\\
       5&0&1&1&0&1&0&1&0&0&1&0 &0\\
       6&0&0&1&0&0&1&0&1&0&0&1 &0\\
       7&0&0&0&1&0&0&1&0&0&0&0 &1\\
       8&0&0&0&0&1&0&0&0&0&1&0 &0\\
       9&0&0&0&0&0&1&0&0&1&0&1 &0\\
      10&0&0&0&0&0&0&1&0&0&1&0 &1\\
      11&0&0&0&0&0&0&0&1&0&0&1 &0\\
    \end{tabular}
  \end{center}
  \caption{A graph and its adjacency matrix.}
  \figlabel{graph-adj}
\end{figure}

隣接行列で効率がよくないのは#outEdges(i)#と#inEdges(i)#である。
これを実装するためには、#a#における対応する行または列の#n#個のエントリを順に見て、各添え字#j#についてそれぞれ#a[i][j]#と#a[j][i]#が真かどうかを確認しなければならない。
\javaimport{ods/AdjacencyMatrix.outEdges(i).inEdges(i)}
\cppimport{ods/AdjacencyMatrix.outEdges(i,edges).inEdges(i,edges)}
これらの操作は明らかに$O(#n#)$の時間がかかる。

隣接行列による表現のもうひとつの短所は、これが大きいことである。

$#n#\times #n#$の真偽値の行列を格納するには$#n#^2$ビット以上のメモリが必要である。
真偽値を単純にならべた行列では実際は$#n#^2$バイトのメモリを使う。
より手の込んだ実装で、#w#個の真偽値をワードに詰め込めば、領域使用量は$O(#n#^2/#w#)$ワードのメモリに減らせる。

\begin{thm}
#AdjacencyMatrix#は#Graph#インターフェースを実装する。
#AdjacencyMatrix#は次の操作をサポートする。
\begin{itemize}
  \item #addEdge(i,j)#・#removeEdge(i,j)#・#hasEdge(i,j)#を定数時間で実行できる。
  \item #inEdges(i)#・#outEdges(i)#を時間$O(#n#)$で実行できる。
\end{itemize}
#AdjacencyMatrix#の領域使用量は$O(#n#^2)$である。
\end{thm}

メモリ使用量の多さと#inEdges(i)#・#outEdges(i)#の性能の低さにもかかわらず、#AdjacencyMatrix#が有向な場合もある。
具体的にはグラフ$G$が\emph{密}なとき、つまり辺の数が$#n#^2$に近く、メモリ使用量$#n#^2$が許容できる場合である。

#AdjacencyMatrix#が広く使われるのは、グラフ#G#の性質を計算するための行列#a#の代数的な操作を効率的に実行できるからでもある。
これはアルゴリズムの授業のトピックだが、ここでもひとつだけそのような性質を挙げる。
#a#のエントリを整数(#true#が1、#false#が0)であると見なして、#a#同士の積を行列の掛け算を使って計算すると、行列$#a#^2$が求まる。
積の定義から、次の関係を思い出してほしい。
\[
    #a^2[i][j]# = \sum_{k=0}^{#n#-1} #a[i][k]#\cdot #a[k][j]#
\]
この和をグラフ$G$の文脈で解釈すると、これは$g$が辺#(i,k)#と辺#(k,j)#を共に持つ頂点$#k#$の個数を数えている。
つまりこれは$#i#$から$#j#$への(中間頂点$#k#$を通る)経路であって、長さがちょうど2であるものの個数である。
この観察は、$G$におけるすべての頂点の対についての最短経路を$O(\log #n#)$回だけの行列の積で計算するアルゴリズムの基礎になっている。

\section{#AdjacencyLists#:リストの集まりとしてのグラフ}
\seclabel{adjacency-list}

\index{adjacency list}%
グラフの\emph{隣接リスト}表現は辺を重視するアプローチである。
隣接リストの実装方法は色々ありうる。
この節では単純なものを説明する。
そしてこの節の最後に別のやり方について述べる。
隣接リスト表現ではグラフ$G=(V,E)$はリストの配列#adj#で表現される。
リスト#adj[i]#は頂点#i#と隣接するすべての頂点を含む。
つまり、$#(i,j)#\in E$を満たす添え字#j#をすべて含むのである。
\codeimport{ods/AdjacencyLists.adj.n.AdjacencyLists(n0)}
(例を\figref{graph-adjlist}に示す)
この実装ではリスト#adj#は#ArrayStack#\cpponly{のサブクラス}としている。
なぜなら添え字を使って定数時間で要素にアクセスしたいからである。
他の選択肢もありうる。
特に#adj#を#DLList#として実装してもよいだろう。

\begin{figure}
  \begin{center}
    \includegraphics[scale=0.90909]{figs/graph} \\[3ex]
    \begin{tabular}{|c|c|c|c|c|c|c|c|c|c|c|c|c|}\hline
        0&1&2&3&4&5&6 &7 &8&9 &10&11 \\\hline
        1&0&1&2&0&1&5 &6 &4&8 &9 &10 \\
        4&2&3&7&5&2&2 &3 &9&5 &6 &7 \\
         &6&6& &8&6&7 &11& &10&11& \\
         &5& & & &9&10&  & &  &  & \\
         & & & & &4&  &  & &  &  & \\
    \end{tabular}
  \end{center}
  \caption{A graph and its adjacency lists}
  \figlabel{graph-adjlist}
\end{figure}

#addEdge(i,j)#はリスト#adj[i]#に#j#を加えるだけだ。
\codeimport{ods/AdjacencyLists.addEdge(i,j)}
これは定数時間で実行できる。

#removeEdge(i,j)#はリスト#adj[i]#から#j#を見つけ、それを削除する。
\codeimport{ods/AdjacencyLists.removeEdge(i,j)}
これは$O(\deg(#i#))$の時間がかかる。
ここで$\deg(#i#)$($#i#$の\emph{次数})は$E$のうち$#i#$から出ているものの個数である。
\index{degree}%

#hasEdge(i,j)#も同様だ。
リスト#adj[i]#から#j#を探して、見つかれば真を、そうでないなら偽を返す。
\codeimport{ods/AdjacencyLists.hasEdge(i,j)}
これにかかる時間は$O(\deg(#i#))$である。

#outEdges(i)#は単純である。
\pcodeonly{これはリスト#adj[i]#を返す。}
\javaonly{これはリスト#adj[i]#を返す。}
\cpponly{これはリスト#adj[i]#中身を出力リストにコピーする。}
\pcodeimport{ods/AdjacencyLists.outEdges(i)}
\javaimport{ods/AdjacencyLists.outEdges(i)}
\cppimport{ods/AdjacencyLists.outEdges(i,edges)}
\javaonly{これは定数時間で実行できる}
\cpponly{これにかかる時間は$O(\deg(#i#))$である。}

#inEdges(i)#はもう少しタイヘンだ。
すべての頂点$j$について#(i,j)#が存在するかどうか確認し、もしそうなら#j#を出力リストに追加する。
\pcodeimport{ods/AdjacencyLists.inEdges(i)}
\javaimport{ods/AdjacencyLists.inEdges(i)}
\cppimport{ods/AdjacencyLists.inEdges(i,edges)}
この操作は非常に時間がかかる。
すべての頂点の隣接リストを見て回る必要があるので、$O(#n# + #m#)$の時間がかかる。

次の定理は上で説明したデータ構造の性能をまとめたものである。

\begin{thm}
#AdjacencyLists#は#Graph#インターフェースを実装する。
#AdjacencyLists#は次の操作をサポートする。
\begin{itemize}
  \item #addEdge(i,j)#は定数時間で実行できる。
  \item #removeEdge(i,j)#・#hasEdge(i,j)#にかかる時間は$O(\deg(#i#))$である。
  \javaonly{\item #outEdges(i)#は定数時間で実行できる。}
  \cpponly{\item #outEdges(i)#にかかる時間は$O(\deg(#i#))$である。}
  \item #inEdges(i)#にかかる時間は$O(#n#+#m#)$である。
\end{itemize}
#AdjacencyLists#の領域使用量は$O(#n#+#m#)$である。
\end{thm}

先程少し言ったように、グラフを隣接リストとして実装する方法には色々ある。
いくつか気になることがあるだろう。
\begin{itemize}
  \item #adj#の要素を格納するにはどんなデータ構造を使うのがいいだろう。
  配列ベースのもの、ポインタベースのもの、あるいはハッシュテーブルだろうか。
  \item 任意の#i#について$#(j,i)#\in E$を満たす#j#のリストであ 二次隣接リスト#inadj#があるべきだろうか。
  これは#inEdges(i)#の実行時間を劇的に改善するが、辺を追加・削除する際の仕事をを少し増やす。
  \item #adj[i]#における辺#(i,j)#は対応する#inadj[j]#のエントリへの参照を持つべきだろうか。
  \item 辺は一級オブジェクトであるべきだろうか。
  このとき#adj#はは頂点のリストではなく、辺のリストを持つことになる。
\end{itemize}
これらの選択肢の大部分は、実装の複雑さと性能とのトレードオフをふまえて考えることになる。

\section{グラフの走査}

この節ではグラフの頂点#i#からはじめて、#i#から到達可能なすべての頂点を探索するアルゴリズムをふたつ紹介する。
いずれも場合も隣接リストで表現されたグラフを使うのが適切である。
よって、この節でアルゴリズムを分析するときにはグラフの表現が#AdjacencyLists#であることを仮定する。

\subsection{幅優先探索}

\index{breadth-first-search}%
\emph{幅優先探索}を頂点#i#からはじめると、まずは#i#に隣接する頂点を訪問し、続いて#i#の隣の隣、続いて#i#の隣の隣の隣、というように進んでいく。

このアルゴリズムは二分木における幅優先の走査アルゴリズム(\secref{bintree:traversal})の一般化であって、非常に似ている。
#i#だけをキュー#q#を使う。
#q#から要素を取り出し、取り出した要素に隣接する要素を#q#に追加する。
ここで、追加する要素はまだこれまで#q#に追加していないものであるとする。
木とグラフにおける幅優先探索アルゴリズムの大きな違いは、グラフの場合には同じ頂点を#q#に二度以上追加しないよう気をつける必要があることである。
このためには真偽値の補助配列#seen#を使って、どの頂点が既に見つかっているかを覚えておけばよい。
\codeimport{ods/Algorithms.bfs(g,r)}
\figref{graph}において#bfs(g,0)#を実行する様子の一例を\figref{graph-bfs}に示した。
異なる処理の仕方もありえる。
これは隣接リストの並び順によって決まる。
\figref{graph-bfs}では\figref{graph-adjlist}の隣接リストを使った。

\begin{figure}
  \begin{center}
    \includegraphics[scale=0.90909]{figs/graph-bfs}
  \end{center}
  \caption[Breadth-first-search]{An example of breadth-first-search starting at node 0. Nodes are
  labelled with the order in which they are added to #q#.  Edges that
  result in nodes being added to #q# are drawn in black, other edges
  are drawn in grey.}
  \figlabel{graph-bfs}
\end{figure}

#bfs(g,i)#の実行時間の解析は簡単である。
#seen#によって同じ頂点は#q#に二度以上追加されることはない。
#q#に頂点の追加する(そして後で削除する)処理は定数時間で実行でき、合計$O(#n#)$だけの時間がかかる。
すべての頂点が内部ループにおいて高々一度処理されるので、すべての隣接リストが高々一度処理される。
よって$G$の辺は高々一度だけ処理される。
内部ループが一周すると辺がひとつ処理され、この各周は定数時間で実行できるので、合計$O(#m#)$だけの時間がかかる。
以上より、アルゴリズム全体の実行時間は$O(#n#+#m#)$である。

次の定理は#bfs(g,r)#の性能をまとめたものである。
\begin{thm}\thmlabel{bfs-graph}
#AdjacencyLists#で実装された#Graph# #g#を入力すると、#bfs(g,r)#の実行時間は$O(#n#+#m#)$である。
\end{thm}

幅優先の走査には特別な性質がある。
#bfs(g,r)#を呼ぶと#r#からの有向経路が存在するすべての頂点を#q#に追加する。(そしてそれをいつか#q#から取り出す。)
また、#r#から距離0の頂点(#r#自身)は、#r#から距離1の頂点より先に#q#に追加され、距離1の頂点は距離2の頂点よりも先に#q#に追加され、これが繰り返される。
そのため、#bfs(g,r)#は#r#からの距離の昇順で頂点を訪問し、#r#から到達不可能な頂点を訪問することはない。

そのため、幅優先探索の特に便利な応用は最短経路の計算である。
#r#からすべての頂点への最短経路を求めるために、長さ#n#の補助配列#p#を利用する#bfs(g,r)#の変種を使える。
頂点#j#を#q#に追加するとき、#p[j]=i#とする。
こうすると#p[j]#は#r#から#j#への最短経路における、最後から二番目の頂点になる。
#p[p[j]#, #p[p[p[j]]]#...とこれを繰り返すと、#r#から#j#への最短経路を(逆順に)再構築できる。

\subsection{深さ優先探索}

\emph{深さ優先探索}は二分木における標準的な走査アルゴリズムに似ている。
\index{depth-first-search}%
このアルゴリズムではある部分木を完全に探索し終えてから根の方向に戻り、そして別の部分木の探索に進む。
別の考え方をすると、深さ優先探索は幅優先探索に似ていて、その違いはスタックの代わりにキューを使うことである。

深さ優先探索において各頂点#i#には色#c[i]#を割り当てる。
未訪問の頂点は#white#、現在訪問中の頂点は#grey#、既に訪問した頂点は#black#とする。
深さ優先探索は再帰的なアルゴリズムとして考えるのが簡単である。
#r#を訪問するところから処理がはじまる。
頂点#i#を訪問するとき、#i#の色を#grey#にする。
続いて#i#の隣接リストを見て、その中の白い頂点を再帰的に訪問する。
最後に#i#の色を#black#にして、#i#の処理を終える。
\codeimport{ods/Algorithms.dfs(g,r).dfs(g,i,c)}
\figref{graph-dfs}にこのアルゴリズムの処理の例を示す。

\begin{figure}
  \begin{center}
    \includegraphics[scale=0.90909]{figs/graph-dfs}
  \end{center}
  \caption[Depth-first-search]{An example of depth-first-search starting at node 0. Nodes are
  labelled with the order in which they are processed.  Edges that
  result in a recursive call are drawn in black, other edges
  are drawn in #grey#.}
  \figlabel{graph-dfs}
\end{figure}

深さ優先探索のことを考えるのには再帰は便利なのだが、実装する際にはこれは最善の方法ではない。
XXX: stack overflow
上のコードは、stackのoverflowによって大きなグラフの探索に失敗してしまうことがある。
別の実装方法として、再帰を明示的なスタック#s#に置き換えることが考えられる。
次の実装はこれを行ったものである。

XXX: black にしなくていいのか?
\codeimport{ods/Algorithms.dfs2(g,r)}
上のコードでは、次の頂点#i#が処理されるとき、#i#の色を#grey#にし、#i#の隣接行列に入っていた頂点をスタックに積み、次はそのうちの一つを#i#にする。

当然だが#dfs(g,r)#・#dfs2(g,r)#の実行時間は#bfs(g,r)#と同じである。
\begin{thm}\thmlabel{dfs-graph}
#AdjacencyLists#で実装された#Graph# #g#を入力すると、#dfs(g,r)#・#dfs2(g,r)#の実行時間はいずれも$O(#n#+#m#)$である。
\end{thm}

幅優先探索と同様に、深さ優先探索の各実行にもある木を対応づけられる。
頂点$#i#\neq #r#$の色が#white#から#grey#になるのは、ある頂点#i'#を再帰的に処理する中で#dfs(g,i,c)#を呼び出したからである。
(#dfs2(g,r)#の場合は#i#は#i'#をスタックで置き換えた頂点のうちの一つである。)
#i'#を#i#の親だと考えると、#r#を根とする木が得られる。
\figref{graph-dfs}では、この木は頂点0から頂点11への経路である。

深さ優先探索の重要な性質を述べる。
#i#の色が#grey#であるとき、#i#から他の頂点#j#への白い頂点だけを辿る経路が存在する。
そして、#i#の色が#black#になるよりも前に、#j#の色は#grey#、そして#black#になる。
(これは背理法で証明できる。#i#から#j#へのある経路$P$を考えればよい。)

この性質は例えば循環の検出に役立つ。
\index{cycle detection}%
\figref{dfs-cycle}を参照せよ。
#r#から到達可能なある循環$C$があるとする。
#i#を$C$の中で色が#grey#である最初の頂点とし、#j#を$C$において#i#の前にある頂点とする。
このとき上の性質から、#j#の色は#grey#になり、辺#(j,i)#を辿るときにも、#i#の色はまだ#grey#である。
深さ優先探索において#i#から#j#への経路$P$が存在し、一方辺#(j,i)#も存在するので、$P$も循環であることがわかる。

\begin{figure}
  \begin{center}
    \includegraphics[scale=0.90909]{figs/dfs-cycle}
  \end{center}
  \caption[Cycle detection]{The depth-first-search algorithm can be used to detect cycles
  in $G$. The node #j# is coloured #grey# while #i# is still #grey#.  This
  implies that there is a path, $P$, from #i# to #j# in the depth-first-search
  tree, and the edge #(j,i)# implies that $P$ is also a cycle.}
  \figlabel{dfs-cycle}
\end{figure}

